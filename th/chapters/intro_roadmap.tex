%---
\section{Thesis roadmap and Contributions}
%---

The original research presented in this thesis covers topics related to the computational and statistical methodologies related to the analysis of sequencing variants to unveil genetic links to complex disease. 
Broadly speaking, we address three types of problems: 
i) data processing of large datasets from high throughput biological experiments such as resequencing in the context of a GWAS (Chapter \ref{ch:bds}); 
ii) functional annotation of variants, i.e. calculating variant's impact at the molecular, cellular or even clinical level (Chapter \ref{ch:snpeff}); 
iii) identification of genetic risk factors for complex disease using models that combine population-level and evolutionary-level data to detect putative epistatic interactions (Chapter \ref{ch:gwas}). 
When applicable, background material specific to each chapter is presented in a preface, together with an explanation of how that chapter ties in with the rest of the thesis.

This thesis comprises text and figures of articles that have either been published, submitted for publication, or ready to be submitted (waiting upon data embargo restrictions):
\\

\begin{description}
	
	\item[Chapter \ref{ch:bds}] ~ 
	
		\begin{enumerate}
			\item \textbf{P. Cingolani}, R. Sladek, and M. Blanchette. ``BigDataScript: a scripting language for data pipelines." Bioinformatics 31.1 (2015): 10-16.
		\end{enumerate}

		For this paper, PC conceptualized the idea and performed the language design and implementation. RS \& MB helped in designing robustness testing procedures. PC, RS \& MB wrote the manuscript.
		\\
	
	\item[Chapter \ref{ch:snpeff}] ~
	
		\begin{enumerate}[resume]
			\item \textbf{P. Cingolani}, A. Platts, M. Coon, T. Nguyen, L. Wang, S.J. Land, X. Lu, D.M. Ruden, et al. ``A program for annotating and predicting the effects of single nucleotide polymorphisms, snpeff: Snps in the genome of drosophila melanogaster strain $w^{1118}; iso-2; iso-3$". Fly, 6(2), 2012.
		\end{enumerate}

		For this paper, PC conceptualized the idea, implemented the program and performed testing.
		AP contributed several feature ideas, software testing and suggested improvements.
		XL, DR, SL, LW, TN, MC, LW performed mutagenesis and sequencing experiments.
		XL and DR performed the biological interpretation of the data.
		All authors contributed to the manuscript.
		\\

		SnpEff's accompanying publication (SnpSift):
	
		\begin{enumerate}[resume]		
			\item \textbf{P. Cingolani}, V. M. Patel, M. Coon, T. Nguyen, S. Land, D. M. Ruden, and X. Lu.`` Using drosophila melanogaster as a model for genotoxic chemical mutational studies with a new program, snpsift". Toxicogenomics in non-mammalian species, page 92, 2012.
		\end{enumerate}
		
		~ \\

		We used SnpEff \& SnpSift and developed a number of new functionalities in the context of two collaborative GWAS projects on type II diabetes:

		~ \\
			
		\begin{enumerate}[resume]
		
			\item M. McCarthy, T2D Genes Consortia. ``Variation in protein-coding sequence and predisposition to type 2 diabetes", Ready for submission.
			
			\item A. Mahajan, X. Sim, H. Ng, A. Manning, M. Rivas, H. Heather, A. Locke, N. Grarup, H. K. Im, \textbf{P. Cingolani}, et al. ``Identification and Functional Characterization of G6PC2 Coding Variants Influencing Glycemic Traits Define an Effector Transcript at the G6PC2-ABCB11 Locus." PLoS genetics 11.1 (2015): e1004876-e1004876.
		
		\end{enumerate}
		~ \\
	
	\item[Chapter \ref{ch:gwas}] ~
	
		\begin{enumerate}[resume]
		\item \textbf{P. Cingolani}, R. Sladek, and M. Blanchette. ``A co-evolutionary approach for detecting epistatic interactions in genome-wide association studies". Ready for submission (data embargo restrictions).
		\end{enumerate}
	
		For this paper, PC designed the methodology under the supervision of MB and RS. PC implemented the algorithms. PC, RS \& MB wrote the manuscript. This work uses data from the T2D consortia, thus it cannot be published until the main T2D paper is accepted for publication (according to T2D data embargo).
		\\
	
	\item[Other contributions] ~	\linebreak
		During my thesis I have co-authored several other scientific articles (grouped by topic) published, submitted for publication, or ready to be submitted, not mentioned in this thesis:
		\\

	\item[Epigenetics] ~

		\begin{enumerate}[resume]
			\item \textbf{P. Cingolani}, X. Cao, R. Khetani, C.C. Chen, M. Coon, A. Bollig-Fischer, S. Land, Y. Huang, M. Hudson, M. Garfinkel, and others. ``Intronic Non-CG DNA hydroxymethylation and alternative mRNA splicing in honey bees." BMC genomics 14.1 (2013): 666.
			\item M. Senut, A. Sen, \textbf{P. Cingolani}, A. Shaik, S. Land, Susan J and D. M. Ruden. ``Lead exposure disrupts global DNA methylation in human embryonic stem cells and alters their neuronal differentiation." Toxicological Sciences (2014).
			\item D. M. Ruden, \textbf{P. Cingolani}, A. Sen, W. Qu, L. Wang, M. Senut, M. Garfinkel, V. Sollars, X. Lu, ``Epigenetics as an answer to Darwin's 'special difficulty' Part 2: Natural selection of metastable epialleles in honeybee castes", Frontiers in Genetics (2015).
			\item M. Senut, A. Sen, \textbf{P. Cingolani}, A. Shaik, S. Land, Susan J and D. M. Ruden. ``Lead exposure induces changes in 5-hydroxymethylcytosine clusters in CpG islands in human embryonic stem cells and umbilical cord blood", Submitted to `Epigenomics.
			\item M. Senut, \textbf{P. Cingolani}, A. Sen, Arko, A. Kruger, A. Shaik, H. Hirsch, S. Suhr, D. Ruden. ``Epigenetics of early-life lead exposure and effects on brain development." Epigenomics 4.6 (2012): 665-674.
		\end{enumerate}
		~ \\
	
	\item[GWAS \& Disease] ~
	
		\begin{enumerate}[resume]
			\item K. Oualkacha, Z. Dastani, R. Li, \textbf{P. Cingolani}, T. Spector, C. Hammond, J. Richards, A. Ciampi, C. Greenwood. ``Adjusted sequence kernel association test for rare variants controlling for cryptic and family relatedness." Genetic epidemiology 37.4 (2013): 366-376.
			\item S. Bongfen, I. Rodrigue-Gervais, J. Berghout, S. Torre, \textbf{P. Cingolani}, S. Wiltshire, G. Leiva-Torres, L. Letourneau, R. Sladek, M. Blanchette, and others. ``An N-ethyl-N-nitrosourea (ENU)-induced dominant negative mutation in the JAK3 kinase protects against cerebral malaria." PloS one 7.2 (2012): e31012.
			\item C. Meunier, L. Van Der Kraak, C. Turbide, N. Groulx, I. Labouba, Ingrid, \textbf{P. Cingolani}, M. Blanchette, G. Yeretssian, A. Mes-Masson, M. Saleh, and others. ``Positional mapping and candidate gene analysis of the mouse Ccs3 locus that regulates differential susceptibility to carcinogen-induced colorectal cancer." PloS one 8.3 (2013): e58733.
			\item G. Caignard, G. Leiva-Torres, M. Leney-Greene, B. Charbonneau, A. Dumaine, N. Fodil-Cornu, M. Pyzik, \textbf{P. Cingolani}, J. Schwartzentruber, J. Dupaul-Chicoine, and others. ``Genome-wide mouse mutagenesis reveals CD45-mediated T cell function as critical in protective immunity to HSV-1." PLoS pathogens 9.9 (2013): e1003637.
			\item M. Bouttier, D. Laperriere, M. Babak Memari, M. Verway, E. Mitchell, \textbf{P. Cingolani}, T. Wang, M. Behr, R. Sladek, M. Blanchette, S. Mader and J. White. ``Genomics analysis reveals elevated LXRα signaling reduces M. tuberculosis viability", Submitted to Journal of Clinical Investigation.
			\item M. Bouttier, D. Laperriere, M. Babak Memari, M. Verway, E. Mitchell, \textbf{P. Cingolani}, T. Wang, M. Behr, R. Sladek, M. Blanchette, S. Mader and J. White. ``Genomic analysis of enhancers engaged in M. tuberculosis-infected macrophages reveals that LXR signaling reduces mycobacterial burden", Submitted to PLOS Pathogens.
		\end{enumerate}	
		~ \\
	
	\item[Fuzzy logic] ~

		\begin{enumerate}[resume]
			\item \textbf{P. Cingolani} and Jesus Alcala-Fdez. ``jFuzzyLogic: a robust and flexible Fuzzy-Logic inference system language implementation." FUZZ-IEEE. 2012.
			\item \textbf{P. Cingolani} and Jesus Alcala-Fdez. ``jFuzzyLogic: a java library to design fuzzy logic controllers according to the standard for fuzzy control programming." International Journal of Computational Intelligence Systems (2013), vol 6, pages 65-75.
		\end{enumerate}	

\end{description}
