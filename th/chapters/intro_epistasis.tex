%---
\section{Epistasis \label{sec:epi}}
%---

%In this section we introduced the basic concepts and methodologies used in GWAS. Although fairly mature, there is still heavy research and continuous improvement on GWAS statistical methods. Not only it is well known that traditional (i.e. single marker) GWAS methods fail under non-additive models \cite{culverhouse2002perspective}, but also variants so far discovered using these methods do not account for all the expected phenotypic variance attributed to genetic causes (i.e. missing heritability). As other authors pointed out \cite{cordell2009detecting, zuk2012mystery, zuk2014searching}, this might be because we need to look for epistatic variants which are not taken into account using these methods. In the next section, and in Chapter \ref{ch:gwas}, we cover the topic of epistatic GWAS analysis.

At the beginning of the $20^{th}$ century some deviations form classical Mendelian inheritance were characterized.
William Bateson first described epistasis in 1907 \cite{tyler2009shadows} assessing a discrepancy between the prediction of segregation ratios assuming individual genes and the real outcome \cite{phillips2008epistasis}.
The term epistasis literally means ``standing upon" was used to describe ``characters'' layered on top of other each other thus masking their expression. 
Reflecting this original definition, nowadays the term epistasis is used to describe an allele at one locus masks the expression of another allele at a different locus \cite{cordell2002epistasis}.
The way epistasis was used to describe the situation in which the actions of one locus mask the allelic effects of another locus, is an extension of dominance where a completely dominant alleles mask the effects of the recessive allele at the same locus. \cite{carlborg2004epistasis, cordell2002epistasis}.

The concept of epistasis is often interpreted as mutations in two genes producing a phenotype that is surprising considering the individual effect of each mutation and can point to functional relationships between genes and pathways \cite{mani2008defining}.
Epistasis, can be used as a tool for understanding the genetic pathways' structure and function as well as evolutionary dynamics \cite{phillips2008epistasis}. 
Some authors even relate the analysis of gene interaction patterns to the fundamentals of systems biology \cite{phillips2008epistasis}.

The term epistasis has expanded to describe many complex interactions among genetic loci  \cite{phillips2008epistasis}.
Geneticists used epistasis to describe different things: 
\begin{itemize}
	\item Functional epistasis: The molecular interactions that proteins. Usually these interactions consist of proteins within the same pathway or of within a complex \cite{phillips2008epistasis}
	\item Compositional epistasis: Describes the traditional usage of epistasis as described by Bateson (i.e. blocking of one allelic effect by an allele at another locus)  \cite{phillips2008epistasis}.
	\item Statistical epistasis: This terminology is attributed to Fisher defined as a deviance from genetic additive effects, this essentially treats it as a residual term in genetic analysis \cite{zhao2006test}.
\end{itemize}

Epistasis can be classified by the way a deviation of a double-mutant organism's phenotype differed from the expected neutral phenotype\cite{mani2008defining}.
An interaction is known as ``synergistic" or ``synthetic" when the double mutant has a more extreme phenotype than expected
When the phenotype is less severe than expected, then there is a ``diminishing returns" or ``alleviating" interaction, this is often attributed to gene products operating in series within the pathway. 
A typical example is a mutation in one gene impairing a whole pathway, thus masking the consequence of mutations in other genes of the same pathway. \cite{mani2008defining}.

Often, the phenotype in human genetics is qualitative and dichotomous, indicating presence or absence of disease. \cite{cordell2002epistasis}.
Thus mathematical models calculating the joint action of more than one loci focus on the penetrance (the probability of developing disease given genotype).
Assuming an allele is required at both loci in order to express the trait, the effect of allele A can only be observed when allele B is also present.
This means that the effect at locus A appears masked by locus B and vice-versa \cite{cordell2002epistasis}, which is not precisely analogous to
what Bateson descrived. 
In Bateson's definition, if factor B is epistatic to factor A, then factor A is not expected to be epistatic to factor B also.  \cite{cordell2002epistasis}
Four mathematically different definitions of interaction have been used (namely Product, Additive, Log, and Min) \cite{mani2008defining}, but even though some definitions yield identical results under some conditions, an alternative definition choice can lead to different consequences\cite{mani2008defining}.

Defining interaction requires measuring phenotype and a neutrality.
Neutrality function predicts the phenotype of an organism without interacting mutations. 
Fitness, is central phenotype measurement to many large-scale genetic interaction studies, it can be defined by population allele frequencies or using growth rates of microbial cultures \cite{mani2008defining}. 
Different measures of fitness can be used in epistasis: i) exponential growth rate of mutant strain respect to wild type ; ii) the increase in population in one wild-type generation; and iii) the relative number of progeny (in one wild-type generation) \cite{mani2008defining}.

Genetic interaction studies have also differed in their choice of neutrality functions, generally using either a multiplicative or a minimum mathematical function. Multiplicative function predicts fitness to be the product of the corresponding single-mutant fitness values. This multiplicative function can be used with the three aforementioned fitness measures to obtain three different definitions of genetic interaction \cite{mani2008defining}.

The ``Min" definition of genetic interaction is simply the minimum neutrality of the expected results form non-interacting mutations (e..g the fitness of the less-fit mutant). 
All the above fitness measurement yield the same set of genetic interactions under this definition. 
For example if each mutation disrupts a distinct pathway limiting cell growth in a way that one mutation is substantially more limiting than the other, the double mutant might is expected have same result as the most-limiting single mutant \cite{mani2008defining}.

It has been shown choice of definition can dramatically alter the resulting set of interactions \cite{mani2008defining}.
To evaluate this the authors in \cite{mani2008defining} applied all four definitions to two studies providing quantitative growth-rate measurements of cell populations. 
They show that: 
i) additive and Log definitions have different biases; 
ii) Product and Log definitions are equivalent for deleterious mutations; 
iii) the product definition can reveal functional relationships missed by the Min definition; 
and iv) interaction networks from Min and Product definitions differ greatly. 
This leads to the question on which definition to use. 
By examining the deviation distribution of expected (double-mutant) phenotype from the observed phenotype they found that Product and Log definitions not only are the closest to the ideal, but also are practically equivalent when single mutants are deleterious \cite{mani2008defining}.

%---
% Epistasis history and definition
%---

%William Bateson first described epistasis in 1907.(2) Like pleiotropy, this concept was developed to explain deviations from Mendelian inheritance The term literally means ``standing upon", and Bateson used it to describe characters that were layered on top of other characters thereby masking their expression.  \cite{tyler2009shadows}
%The commonly used definition of epistasis--an allele at one locus masks the expression of an allele at another locus--reflects this original definition. \cite{tyler2009shadows}
%
%The term `epistasis' was initially used in the context of Mendelian inheritance; environmental effects are relatively unimportant for Mendelian traits, so Ii individuals can be clearly assigned to one of a limited number of classes according to their phenotype. 
%Here, epistasis was used to describe the situation in which the actions of one locus mask the allelic effects of another locus, in the same way that completely dominant alleles mask the effects of the recessive allele at the same locus. \cite{carlborg2004epistasis}
%
%The term `epistatic' was first used in 1909 by Bateson (1) to describe a masking effect whereby a variant or allele at one locus (denoted at that time as an `allelomorphic pair') prevents the variant at another locus from manifesting its effect.  \cite{cordell2002epistasis}
%This was seen as an extension of the concept of dominance. 
%There are, however, some problems with this definition, particularly when applied to binary traits. In human genetics, the phenotype of interest is often qualitative and usually dichotomous, indicating presence or absence of disease. \cite{cordell2002epistasis}
% Mathematical models for the joint action of two or more loci usually focus on the penetrance, the probability of developing disease given genotype. \cite{cordell2002epistasis}
%Suppose that a predisposing allele is required at both loci in order to exhibit the trait, i.e. one or more copies of both allele A and allele B are required. Then, when the effects of both loci are considered, we obtain the penetrance table shown in Table 2. In this table, the effect of allele A can only be observed when allele B is also present: without the presence of B, the effect of A is not observable. The effect at locus A would appear to be `masked' by that at locus B. \cite{cordell2002epistasis}
%This leads to a situation that is not precisely analogous to that described by Bateson (1). In Bateson's (1) definition, it is clear that if factor B is epistatic to factor A, we do not expect factor A to also be epistatic to factor B.  \cite{cordell2002epistasis}
%Table 3 is usually assumed to correspond to a situation in which the biological pathways involved in disease influenced by the two loci are at some level separate or independent (5). \cite{cordell2002epistasis}
%
%The use of the term epistasis has since expanded to describe nearly any set of complex interactions among genetic loci  \cite{phillips2008epistasis}
%Over the years geneticists have used epistasis to describe three distinct things: the functional relationship between genes, the genetic ordering of regulatory pathways and the quantitative differences of allele-specific effects \cite{phillips2008epistasis}
%Over the years the disparate needs of geneticists have led to a plethora of differently nuanced meanings for the term epistasis, all of which involve gene interactions at various levels  \cite{phillips2008epistasis}
%`Functional epistasis' addresses the molecular interactions that proteins (and other genetic elements) have with one another, whether these interactions consist of proteins that operate within the same pathway or of proteins that directly complex with one another18 \cite{phillips2008epistasis}
%`Compositional epistasis' is a new term that is intended to describe the traditional usage of epistasis as the blocking of one allelic effect by an allele at another locus.  \cite{phillips2008epistasis}
%`statistical epistasis' is the usage of epistasis that is attributed to Fisher (BOX 1), in which the average deviation of combinations of alleles at different loci is estimated over all other genotypes present within a population.  \cite{phillips2008epistasis}
%
%It should be apparent that the global analysis of geneinteraction patterns bears a striking resemblance to what is now called systems biology \cite{phillips2008epistasis}
%
%often been defined as a deviance from genetic additive effects, which is essentially treated as a residual term in genetic analysis and leads to low power in detecting the presence of interacting effects \cite{zhao2006test}

%Sometimes mutations in two genes produce a phenotype that is surprising in light of each mutation's individual effects. This phenomenon, which defines genetic interaction, can reveal functional relationships between genes and pathways. \cite{mani2008defining}
%Recent studies have used four mathematically distinct definitions of genetic interaction (here termed Product, Additive, Log, and Min). Whether this choice holds practical consequences has not been clear, because the definitions yield identical results under some condition \cite{mani2008defining}
%Here, we show that the choice among alternative definitions can have profound consequences. \cite{mani2008defining}
%
%A quantitative genetic interaction definition has two components: a quantitative phenotypic measure and a neutrality function that predicts the phenotype of an organism carrying two noninteracting mutations. Interaction is then defined by deviation of a double-mutant organism's phenotype from the expected neutral phenotype \cite{mani2008defining}
%A double mutant with a more extreme phenotype than expected defines a synergistic (or synthetic) interaction between the corresponding mutations (synthetic lethality, in the extreme case). \cite{mani2008defining}
%Alleviating or ``diminishing returns" interactions, in which the double-mutant phenotype is less severe than expected, often result when gene products operate in concert or in series within the same pathway. Alleviating interactions arise, for example, when a mutation in one gene impairs the function of a whole pathway, thereby masking the consequence of mutations in additional members of that pathway. \cite{mani2008defining}
%One class of phenotype, fitness, has been central to many large-scale genetic interaction studies. Although fitness was originally measured in terms of population allele frequencies (1, 22, 23), it can also be measured by using growth rates of isogenic microbial cultures. \cite{mani2008defining}
%Genetic interaction studies have used different measures of fitness, including: (i) the exponential growth rate of the mutant strain relative to that of wild type (4, 9, 15, 19) (the relative-growthrate measure); (ii) the increase in mutant population relative to wild type in one wild-type generation (the relative-population measure) (6); and (iii) the number of progeny per mutant organism relative to the number of progeny for wild type in one wild-type generation (the relative-progeny measure) (24) \cite{mani2008defining}
%Genetic interaction studies have also differed in their choice of neutrality functions, generally using either a multiplicative or a minimum mathematical function. \cite{mani2008defining}
%
%The multiplicative function, which was originally applied to fitness measures defined in terms of allele frequencies, predicts double-mutant fitness to be the product of the corresponding single-mutant fitness values. The multiplicative function can be combined with each of the three fitness measures above to yield three distinct definitions of genetic interaction (4, 6, 15, 19, 24). \cite{mani2008defining}
%A fourth (Min) definition of genetic interaction results from the minimum neutrality function, under which noninteracting mutations are expected to yield the fitness of the less-fit single mutant. Each fitness measure above yields an identical set of genetic interactions under this function. A hypothetical example illustrates one rationale for the Min definition: Two single mutations each disrupt a distinct cellular pathway that limits cell growth, such that one of these mutations is substantially more limiting than the other. The double mutant might then be expected to exhibit the phenotype of the most-limiting single mutant.  \cite{mani2008defining}
%It has not been clear whether the choice of genetic interaction definition has any practical consequences. To evaluate the impact of definition choice, we applied each of the four definitions in turn to two reference studies. \cite{mani2008defining}
%Here, we show that the choice of definition can dramatically alter the resulting set of genetic interactions and the extent to which they correspond to shared gene function.  \cite{mani2008defining}
%For a gene pair (x, y), we refer to the fitness of the two single mutants and the double mutant, respectively, as Wx, Wy, and Wxy. \cite{mani2008defining}
%The neutrality function E(Wxy), predicting double-mutant fitness for a strain with mutations in noninteracting genes x and y, is defined differently under the Min, Product, Log, and Additive  \cite{mani2008defining}
%
%DATASET: To evaluate the impact of definition choice, we applied each of the four definitions in turn to two reference studies, St. Onge et al. (19) (Study S) and Jasnos and Korona (6) (Study J), both providing quantitative growth-rate measurements of isogenic wild-type and singleand double-mutant cell populations. \cite{mani2008defining}
%RESULTS: The Choice of Genetic Interaction Definition Matters: \cite{mani2008defining}
%Additive and Log Definitions Demonstrate Different Biases: However, we had observed that interaction strength had a significant positive bias (under all definitions) for pairs involving mutations with extreme fitness effects. \cite{mani2008defining}
%Product and Log Definitions Are Equivalent for Deleterious Mutations:  \cite{mani2008defining}
%The Product Definition Reveals Functional Relationships Missed by the Min Definition. \cite{mani2008defining}
%Genetic Interaction Networks from Min and Product Definitions Differ Greatly. \cite{mani2008defining}
%
%WHICH DEFINITION TO USE?: We examined the distribution of 􏰍, the deviation of the expected double-mutant phenotype from the observed double mutant phenotype, and found the Product and Log definitions to be closest to this ideal in general. Additionally, we showed that the Log and Product definitions are practically equivalent when both single mutants are deleterious. \cite{mani2008defining}

%---
% QTL
%---
The presence or knock-out of a gene are extreme aspects of "perturbation in a complex system", but there are no reasons to expect all forms of epistasis to follow this pattern \cite{phillips2008epistasis}.
When applied to quantitative traits, epistasis also describes a situation in which the phenotype cannot be predicted by the sum of its single-locus component \cite{carlborg2004epistasis}.
Many epistatic QTL interactions have been detected in model organisms leading to the conclusion that epistasis makes a large contribution to the genetic regulation of complex traits  \cite{carlborg2004epistasis}.

%In the case of QUANTITATIVE TRAITS, epistasis describes the general situation in which the phenotype of a given genotype cannot be predicted by the sum of its component single-locus effects1 \cite{carlborg2004epistasis}
%Epistatic QTL-mapping studies in model organisms have detected many new interactions and have therefore concluded that epistasis makes a large contribution to the genetic regulation of complex traits.  \cite{carlborg2004epistasis}
%Complex synthetic interactions. : 
%There is no reason to expect all forms of epistasis to be revealed simply by the absence of a gene, which is certainly an extreme approach to perturbing complex systems. 
%For example, Kroll et al.35 devised a method for looking for interactions that are induced after systematically overexpressing genes. Using this approach, sopko et al.36 found that, when overexpressed in Saccharomyces cerevisiae, about 15\% of a set of 5,280 yeast genes induced a growth defect, with most of the overexpression effects not matching the phenotypes of their corresponding deletions.  \cite{phillips2008epistasis}


\subsection{Epistasis is ubiquitous}

Epistasis is defined as departure from additive effects.
Nevertheless, there is no reason to think that traits should be additive based on a purely biological perspective \cite{zuk2012mystery} since biology is riddled with non-linearity such as genetic networks exhibit binary states, ligand - receptors concentration having sigmoid-like response, concentration saturations of substrate - enzymes reactions, sharp transitions created by cooperative protein binding, the pathways constrained by rate-limiting inputs, etc. \cite{zuk2012mystery}. 
It has been asserted that epistatic effects are not isolated events, but ubiquitous \cite{tyler2009shadows} and probably inherent properties of biomolecular networks.
This leads to think that epistasis in the classical sense may be ubiquitous, a thought which has been partially confirmed from mutational studies \cite{phillips2008epistasis}.
Genetic studies of synthetic traits, which occur only when multiple loci or pathways are all disrupted, in model organisms have identified instances of interacting genes revealing that epistasis may be pervasive \cite{zuk2012mystery}. 
Researchers found \cite{phillips2008epistasis} that when looking for interactions induced by systematically over-expressing genes in Saccharomyces cerevisiae about $15\%$ of studied genes induced growth defects with most over-expression not matching the phenotypes of individual deletions.  

%Researchers found \cite{phillips2008epistasis} that when looking for interactions induced by systematically over-expressing genes in Saccharomyces cerevisiae about $15\%$ of studied genes induced growth defects with most over-expression not matching the phenotypes of individual deletions.  
%From mutational studies we know that epistasis in the classical sense is ubiquitous because genes interact in hierarchical systems to generate biological function.  \cite{phillips2008epistasis}
% From a biological standpoint, there is no a priori reason to expect that traits should be additive. Biology is filled with nonlinearity: The saturation of enzymes with substrate concentration and receptors with ligand concentration yields sigmoid response curves; cooperative binding of proteins gives rise to sharp transitions; the outputs of pathways are constrained by rate-limiting inputs; and genetic networks exhibit bistable states. \cite{zuk2012mystery}
%Genetic studies in model organisms have long identified specific instances of interacting genes (17). Important examples include synthetic traits (e.g., 18), which occur only when multiple loci or pathways are all disrupted. \cite{zuk2012mystery}
%Studies have begun to reveal that epistasis is pervasive.  \cite{zuk2012mystery}
%We assert that epistasis and pleiotropy are not isolated occurrences, but ubiquitous and inherent properties of biomolecular networks. \cite{tyler2009shadows}

\subsection{Epistasis examples: Non-human}

Several genotype-phenotype patterns are known to be caused by epistasis, classic examples include \cite{carlborg2004epistasis}:
coat colour in various animals, 
comb type in chickens, 
kernel colour in wheat,
eye color in flies,
and the h/h blood group (also known as Oh or the Bombay phenotype) in the ABO blood-group in humans.

Coat colour in mammals has been one of the most common examples. 
In pig, the dominant allele at the KIT locus confers white color coat and is dominant over all locus conferring darker color (melanocortin 1 receptor or MC1R). 
This can be determined in individuals with the recessive KIT genotype showing what was classically termed `dominant epistasis', 
yielding a non-Mendelian segregation ratio of 9:4:3 (instead of 9:3:3:1) \cite{carlborg2004epistasis, phillips2008epistasis}.

Drosophila provides another classic example with eye color determination.
Drosophila eye pigmentation scarlet, brown, and white is determined by the synthesis of two drosopterins:  brown pigments (from tryptophan) and red pigments (from GTP) \cite{tyler2009shadows:REF}.
A mutation that prevents production of the brown pigment results in a fly with red eyes and a mutation preventing red pigment results in a fly with brown eyes.
Flies with a mutation in the white gene, neither red nor brown pigment can be synthesized resulting in a fly with white eyes (regardless of the genotype at the brown or scarlet loci) \cite{tyler2009shadows}.

Dozens of quantitative traits indicating strong epistasis in mouse and rat \cite{zuk2012mystery:REF22} by analysing a panel of chromosome substitution strains where the effects attributed to the donor chromosomes exceeds by a median eightfold the expected effect of the donor genome.

Genetic interaction have been study in a systematic and large-scale manner in Saccharomyces cerevisiae \cite{mani2008defining:REF}.
Analysis of quantitative traits loci (QTL) for transcripts levels in a two strain cross demonstrated epistatic interaction for $67\%$ of studied pairs (first the strongest QTL was found and then the strongest remaining QTL conditional on the first genotype was selected) \cite{zuk2012mystery:REF}. 

In a study comparing three Drosophila inbred lines (Drosophila melanogaster Genetic Reference Panel -DGRP) and a large outbred and intercross derived  population \cite{huang2012epistasis}a set of candidate SNPs was selected by assesing allele frequency changes between the extremes of the distribution for each trait. 
The researchers found that the majority of these SNPs participated in at least one epistatic interaction \cite{huang2012epistasis}.
Using this information from epistatic interacting loci they were able to infer networks affecting quantitative traits. \cite{huang2012epistasis}.

%In the case of quantitative genetic variation, several or many genes of largely unknown function combine with environmental influences to control trait variation. This is the case for many complex traits that are of medical relevance in humans or of economic importance in plants and livestock. \cite{carlborg2004epistasis}
%A clear example of this can be seen  [in Fig A]  which the dominant allele (I) at the KIT locus, which confers white-coat colour in the pig, is dominant over all alleles at the MC1R locus (E), which confer a darker coat colour. The effects of the various alleles at the E locus can only be determined in individuals with the recessive genotype ii at the I locus. This example was classically termed `dominant epistasis', which gives a segregation ratio of 12:3:1 for white:black:brown, respectively \cite{carlborg2004epistasis}
%Table 1. Example of phenotypes (e.g. hair colour) obtained from different genotypes at two loci interacting epistatically, under Bateson's (1909) definition of epistasis \cite{cordell2002epistasis}
%Coat colour variation in mammals has long been is one of the most fruitful examples in the study of the relationship between genotype and phenotype. ... epistasis arises when the effects of alleles at one locus are blocked by the presence of a specific allele at another locus. For example, a cross between agouti and extension (now called the melanocortin 1 receptor or Mc1r) double heterozygotes (AaEa) yields the non-Mendelian segregation ratio of 9:4:3 (instead of 9:3:3:1) \cite{phillips2008epistasis}

%In the yeast Saccharomyces cerevisiae, Brem et al. (19) analyzed as quantitative traits the levels of gene transcripts in segregants of a cross between two strains. For each transcript, they found the strongest quantitative trait locus (QTL) in the cross and then, conditional on the genotype at this locus, identified the strongest remaining QTL. In 67\% of cases, these two QTLs demonstrated epistatic interactions. In bacteria, Khan et al. (20) and Chou et al. (21) have recently demonstrated clear epistasis among collections of five mutations that increase growth rate.  \cite{zuk2012mystery}

%In mouse and rat, Shao et al. (22) analyzed a panel of chromosome substitution strains, with each strain carrying a different chromosome from a donor strain on a common recipient genetic background. For dozens of quantitative traits, the sum of the effect attributable to the individual donor chromosomes far exceeds (median eightfold) the total effect of the donor genome, indicating strong epistasis.  \cite{zuk2012mystery}
%An example in insects is the abnormal-abdomen phenotype in Drosophila mercatorum (DeSalle and Templeton 1986; Hollocher et al. 1992; Hollocher and Templeton 1994). \cite{culverhouse2002perspective}
%The study of genetic interaction has become increasingly systematic and large-scale, especially in the yeast Saccharomyces cerevisiae (6, 8-21). \cite{mani2008defining}
%Eye color determination in Drosophila provides a classic example. The genes scarlet, brown, and white, play major roles in a simplified model of Drosophila eye pigmentation. Eye pigmentation in Drosophila requires the synthesis and deposition of both drosopterins, red pigments synthesized from GTP, and ommochromes, brown pigments synthesized from tryptophan. A mutation in brown prevents production of the bright red pigment resulting in a fly with brown eyes, and a mutation in scarlet prevents production of the brown pigment resulting in a fly with bright red eyes. In a fly with a mutation in the white gene, neither pigment can be produced, and the fly will have white eyes regardless of the genotype at the brown or scarlet loci. In this example the white gene is epistatic to brown and scarlet. A mutant genotype at the white locus masks the genotypes at the other loci. \cite{tyler2009shadows}

% NOT ADDED YET!
%Evidence from inbred strains of mice indicates that a quarter or more of the mammalian genome consists of chromosome regions containing clusters of functionally related genes \cite{petkov2005evidence}
%60 genetically diverse inbred strains. \cite{petkov2005evidence}
%forming networks with scale-free architecture. Combining LD data with pathway and genome annotation databases, we have been able to identify the biological functions underlying several domains and networks. \cite{petkov2005evidence}
%As typified by the a and b globin gene clusters, tandem duplications can give rise to gene families whose members develop divergent, but still related, functions over time.  \cite{petkov2005evidence}

%Epistasis-nonlinear genetic interactions between polymorphic loci-is the genetic basis of canalization (the robustness or ability of a population to produce the same phenotype regardless of environmental variability) and speciation, 
%Epistatic interactions can be used to infer genetic networks affecting quantitative traits \cite{huang2012epistasis}
%DATASET: Here, we compared the genetic architecture of three Drosophila life history traits in the sequenced inbred lines of the Drosophila melanogaster Genetic Reference Panel (DGRP) and a large outbred, advanced intercross population derived from 40 DGRP lines (Flyland)\cite{huang2012epistasis}
%Surprisingly, none of the SNPs associated with the traits in Flyland replicated in the DGRP and vice versa. However, the majority of these SNPs participated in at least one epistatic interaction in the DGRP.\cite{huang2012epistasis}
%Our analysis underscores the importance of epistasis as a principal factor that determines variation for quantitative traits and provides a means to uncover genetic networks affecting these traits. \cite{huang2012epistasis}


\subsection{Epistasis examples: Human}

Few instances of epistasis in common human disease have been discovered and well-replicated so far, despite considerable efforts \cite{zuk2012mystery}.
Although many instances of epistasis related to human disease have been published, with examples form coronary artery disease\cite{phillips2008epistasis:REF}, diabetes\cite{phillips2008epistasis:REF}, bipolar effective disorder\cite{phillips2008epistasis:REF} and autism \cite{phillips2008epistasis:REF}; some authors suspect there might be statistical features in the association studies because only a few have functional basis \cite{phillips2008epistasis}.

Perhaps the best examples are interactions involving at least one locus with a large effect such as HLA  \cite{zuk2012mystery}.
Two different interaction involving HLA alleles and ERAP have been discovered in GWAS from ankylosing spondylitis and psoriasis where the HLA alleles have odds ratio of 40.8 and 4.66 respectively \cite{zuk2012mystery:REF}.
In autoimmune disease multiple sclerosis a researchers found evidence of genetic interactions between two histocompatibility loci known to be associated with the disease (HLA-DRB5*0101 in DR2a and HLA-DRB1*1501 in DR2b) \cite{phillips2008epistasis:REF}. 
There was evidence of naturally occurring linkage disequilibrium which is suspected to be generated by strong epistasis \cite{phillips2008epistasis}.
In Type 1 diabetes HLA is assumed to act non-additively with all other risk alleles (HLA has have an effect of 5.5) \cite{zuk2012mystery:REF}.
in Hirschsprung's disease an interaction between RET and EDNRB was uncovered by a genome-wide linkage study (RET having a log-odds of 5.6). \cite{zuk2012mystery:REF}

The ACE gene (angiotensin I converting enzyme) has an epistatic interaction with AGTR1 gene (angiotensin II type 1 receptor ) gene, significantly increasing risk of myocardial infarction when the "D-allele" in ACE  is present in patients carrying a particular AGTR1 allele \cite{carlborg2004epistasis:REF}.

Two different sets of interactions are assumed to be responsible for variation in triglyceride levels.
Notably, the interactions depend on the patient's sex: in females the interactions involves ApoB and ApoE;  and in males the interaction involves the ApoAI/CIII/AIV complex and low-density lipoprotein receptor \cite{culverhouse2002perspective:REF}

Sickle-cell anemia is regarded as a Mendelian trait is modified by epistatic interactions evidenced by the fact that patients homozygous for two polymorphisms near the $G\gamma$ locus have only mild clinical symptoms \cite{culverhouse2002perspective:REF}.

Elevated blood serum cholesterol levels in humans is associated with an ApoE allele depending on the genotype at the LDLR (low density lipoprotein receptor) gene locus \cite{tyler2009shadows:REF}.

%Despite considerable efforts, few well-replicated instances of epistasis in common human disease and trait genetics have been discovered thus far. \cite{zuk2012mystery}
%The only examples to date involve interactions featuring at least one locus with a large marginal e↵ect, such as HLA.  \cite{zuk2012mystery}
%GWAS, in ankylosing spondylitis21 and psoriasis,22 discovered interactions between two di↵erent HLA alleles and ERAP1. (In ankylosing spondylitis, the HLA-B27 allele has an odds ratio of 40.8, and in psoriasis the HLA-C allele has an odds ratio of 4.66.) HLA also plays a role in an interaction e↵ect described in a GWAS of Type 1 diabetes. (In Type 1 diabetes, HLA has a main e↵ect of 5.5, but acts non-additively with the risk of all other alleles considered cumulatively.23). Finally, interaction between RET and EDNRB in Hirschsprung's disease was discovered in a genome-wide linkage study,24 in which RET was strongly associated with disease (log-odds score of 5.6). \cite{zuk2012mystery}
%D-allele of the angiotensin I converting enzyme (ACE) gene and the C-allele of the angiotensin II type 1 receptor (AGTR1) gene3. The risk of myocardial infarction is significantly increased by the ACE D-allele in patients who carry that particular AGTR1 allele. \cite{carlborg2004epistasis}
%There are numerous cases of epistasis appearing as a statistical feature of association studies of human disease. A few recent examples include coronary artery disease63, diabetes64, bipolar effective disorder65 and autism66. Unfortunately, in only a few cases has the functional basis of these potential interactions been revealed.  \cite{phillips2008epistasis}
%One of these cases involves the genetic interactions underlying the autoimmune disease multiple sclerosis. Here, Gregersen et al.67 found evidence that natural selection might be maintaining linkage disequilibrium between the histocompatibility loci HLA-DRB5*0101 (DR2a) and HLA-DRB1*1501 (DR2b) (FIG. 3), which are known to be associated with multiple sclerosis; linkage disequilibrium can be generated by strong epistasis among adjacent loci \cite{phillips2008epistasis}

%Indeed, it has been argued that epistatic interactions are a nearly universal component of the architecture of most common traits. Templeton (2000), for instance, has listed a number of phenotypes in which epistasis plays a large role.  \cite{culverhouse2002perspective}
%In humans, variation in triglyceride levels can be explained, in part, by two sets of interactions: between ApoB and ApoE in females and between the ApoAI/CIII/AIV complex and low-density lipoprotein receptor in males (Nelson et al. 2001) \cite{culverhouse2002perspective}
%Even the seemingly ``simple" Mendelian trait of sickle-cell anemia is revealed to be greatly modified by epistatic interactions. Individuals with sickle-cell anemia who are homozygous for two polymorphisms near the Gg locus (leading to the persistence of fetal hemoglobin) have only mild clinical symptoms \cite{culverhouse2002perspective}
%For example, in humans the E4 allele of apolipoprotein epsilon (ApoE) is associated with elevated blood serum cholesterol levels, but only in individuals with the A2A2 genotype at the low density lipoprotein receptor (LDLR) locus.(3) In other words, the contribution of the ApoE allele to cholesterol levels depends on the genotype at the LDLR locus. \cite{tyler2009shadows}

\subsection{Epistasis and evolution}

From an evolutionary perspective, some authors argue that the nonlinear epistatic interactions between polymorphic loci is the genetic basis of canalization (the robustness or ability of a population to produce the same phenotype regardless of environmental variability) and speciation \cite{huang2012epistasis}.

It has also been pointed out that interactions have an important influence on evolutionary phenomena such as genetic divergence and affects the evolution of the structure of genetic systems \cite{phillips2008epistasis} sine studies have shown that epistasis can have a limiting role on the possible paths that evolution can take \cite{phillips2008epistasis:REF}.
This has been supported by analysis of complex gene regulation patterns in localized genomic regions \cite{phillips2008epistasis:REF}.
For variety of organisms (such as yeast, Caenorhabditis, Drosophila, higher plants, and mammals) genes sharing expression patterns are more likely to be in proximity \cite{petkov2005evidence:REF}.
This evidence shows that regional controls of chromatin structure and expression may give rise to gene clusters by promoting their coregulation \cite{petkov2005evidence}.

Theoretical grounds that date back to Fisher assert that when genes interact there is evolutionary pressure to promote their genetic linkage as a means of enhancing the coinheritance of favorable allelic combinations \cite{petkov2005evidence:REF_FISHER}.
Under this assumption linkage can facilitate the maintenance of epistatic interactions and vice versa, thus explaining how some molecular evolution complexity \cite{phillips2008epistasis}.

%epistasis can have an important influence on a number of evolutionary phenomena, including the genetic divergence between species79, ... the evolution of the structure of genetic systems8 \cite{phillips2008epistasis}
%Thus far, these studies81-85 have shown that epistasis can have a strong role in limiting the possible paths that evolution can take, but not in limiting its eventual outcome. \cite{phillips2008epistasis}
%linkage can facilitate the maintenance of epistatic interactions (and vice versa)86 and could help to explain how molecular complexity evolves \cite{phillips2008epistasis}
%recent analysis of patterns of gene regulation suggest that there can be complex patterns of gene regulation in localized genomic regions8 \cite{phillips2008epistasis}
%Gene clusters may arise as a means of promoting their coregulation through regional controls of chromatin structure and expression, and there is now considerable evidence, well summarized by Hurst et al. [1], that for variety of eukaryotes, including yeast, Caenorhabditis, Drosophila, higher plants, and mammals, genes sharing expression patterns are more likely to be in proximity than would be expected by chance.  \cite{petkov2005evidence}
%...And finally, Fisher [2] and later Nei [3,4] have argued on theoretical grounds that when genes interact epistatically, evolutionary selection will promote their genetic linkage as a means of enhancing the coinheritance of favorable allelic combinations.  \cite{petkov2005evidence}

\subsection{Missing heritability}

At the dawn of the ``GWAS era" in 2002 it was hypothesised that there existed a large class of genetic models for which GWAS would fail, namely purely epistatic models conssiting of models with no additive or dominance variation at any of the susceptibility loci. 
Thus association case/control methods ``will have no power if the loci are examined individually" \cite{culverhouse2002perspective}.
Furthermore, it was mathematically shown that for such models maximizing the broad sense heritability (under some constraints) is equivalent maximizing the interaction variance \cite{culverhouse2002perspective}.

In a seminal series of papers \cite{zuk2012mystery, zuk2014searching} further mathemical prof of the link between epistasis and heritability was provided.
Missing heritability arises by an overestimation of the denominator that happens when epistasis is ignored \cite{zuk2012mystery}.
This overestimatio, called ``phantom heritability", was shown to inflate the denominator over $60\%$ in Cohn's disease, thus could accounting for up to $80\%$ of the missing heritability \cite{zuk2012mystery}.
Even though the prevailing view among geneticists is that interactions play at most a minor role in explaining missing heritability, their works showsthat simple (and plausible) models can give rise to substantial phantom heritability \cite{zuk2012mystery}

In moderately heritable complex diseases for which single-locus analyses have not accounted for the predicted genetic variance these epistatic models provide one possible explanation so it is worth pursuing a hypothesis of interacting loci \cite{culverhouse2002perspective}.

%IN 2002: Thus, for fixed K, p , and p , maximizing the broad AB heritability (h 2 p V /V ) under the constraint repreIT sented by formula (2) is equivalent to the maximizing of VI. \cite{culverhouse2002perspective}. TABLE 2 and 3: Maxima of heritability using epistasis. \cite{culverhouse2002perspective}. Three-locus models can also give rise to higher relative risks than are possible in corresponding two-locus models. Three-locus penetrance models maximizing heritability at the low end of disease prevalence \cite{culverhouse2002perspective}

%missing heritability: overestimation of the denominator happens when epistasis is ignored (phantom) \cite{zuk2012mystery}
%phantom heritability could be 62.8\% in Cohn's disease, thus accounting for 80\% of the current missing heritability \cite{zuk2012mystery}
%Until recently "The prevailing view among human geneticists appears to be that interactions play at most a minor  part in explaining missing heritability."	 \cite{zuk2012mystery}
%But "[they] show that simple and plausible models can give rise to substantial phantom heritability."	 \cite{zuk2012mystery}
%...although the pervasiveness of epistasis in experimental organisms suggests that the true heritability h2 of traits may be much lower than current estimates \cite{zuk2012mystery}

%Researchers of many complex diseases (including non-insulin-dependent diabetes mellitus, prostate cancer, and schizophrenia) face the conundrum of moderately heritable diseases for which locus-by-locus analyses have not accounted for the predicted genetic variance. The models discussed in the present article provide one possible explanation for this. \cite{culverhouse2002perspective}
%These considerations lead us to believe that, in situations in which heritability is moderate to high but in which locus-by-locus analyses do not account for the predicted genetic variance, it is worth pursuing a hypothesis of interacting loci [near the linkage peaks] \cite{culverhouse2002perspective}

\subsection{Detecting Epistasis / interactions}

Linkage disequilibria (LD) between close sites are the result of unrecombined chromosome blocks from common ancestry \cite{REF}, nevertheless LD between widely separated sites suggests epistatic selection forces are at work \cite{REF-FISHER, REF-NATURE, koch2013long}.
In an analysis using Yoruba population (from Ibadan, Nigeria) of the HapMap dataset patterns of LD were quantified and significance of overall disequilibrium per chromosome was evaluated of using randomization \cite{koch2013long}, showing an excess of associations in distant on all of the 22 autosomes. 
Although this is suggestive of epistasis, other hypothesis should not be ruled out:
i) population admixture has been proposed to explain unusual patterns of long range LD \cite{koch2013long:REF}
ii) recombination between distant chromosome blocks may not completely erase LD caused by drift even in a population at demographic equilibrium, 
iii) bottlenecks are particularly effective at generating LD
iv) hitchhiking of linked sites with a positively-selected mutation can generate large haplotype blocks 
v) large inversion and other structural variation alter recombination patterns thus causing LD over unusually large regions \cite{koch2013long:REF}.

Under the assumption that long range LD can hint physical protein interactions the authors of LDGIdb \cite{wang2012ldgidb} created a catalog of over $600,000$ pairs of SNPs showing strong long-range linkage disequilibrium, i.e. pairs of SNP pairs that were either located in different chromosomes or in different LD blocks and had $r^2 \ge 0.8$ \cite{wang2012ldgidb}.
However these simple approaches may be of little utility because of technical issues that must be taken into account when performing such association, since commonly used measures of LD (such as $r^2$ and $D'$) are known to give rise to large linkage when sampling minor allele frequencies (MAF) near 0 \cite{koch2013long}.
A better alternative is to measure the probability that a large value of the disequilibrium $D$ is observed if there is no association further refined by conditioning on the sampled allele frequencies (at the two loci), which has the analytical advantage to asymptotically converge to a Fisher's exact test \cite{koch2013long}.

It is possible to implicitly test the over / under-representation of allele pairs in a given population, i.e. analysis of imbalanced allele pair frequencies \cite{ackermann2012systematic}
The underlying theory is that such allele pairs are under Dobzhansky-Muller incompatibilities which establishes a fitness bias favouring of individuals that inherit over-represented allele combination \cite{ackermann2012systematic}.

The authors in \cite{ackermann2012systematic} studying a population of 2,002 mice in family trios.
They performed a $\chi^2$ test correcting by confounding factors (such as expected frequencies, family structure and allelic drift) based on inspecting $3 x 3$ contingency tables of all possible two-locus allele combinations.
They claim that using their method it is possible to detect more interactions than using independent markers and as a result they were able to identify 168 LD block pairs with imbalanced alleles \cite{ackermann2012systematic}.

By exploiting the  intense selective pressures imposed by the process of inbreeding mice populations it can be expected that clusters of functionally related genes are likely to be selected for coadapted allelic combinations in genes that influence fitness and survival.
This hypothesis would result in regions of linkage disequilibrium (LD) among inbred strain genomes that should occur more often than expected by chance \cite{petkov2005evidence}.
In a study using 60 inbred mouse strains \cite{petkov2005evidence}, the authors study LD using permutation tests showing that extreme patterns of LD give rise to scale-free networks architectures.
Further pathway analysis identifies biological functions underlying several of these networks, hinting that selective factors acting to generate LD networks during inbreeding are a reflect  interaction of functionally \cite{petkov2005evidence}.


%The identity of these strains and the phylogenetic relationships among them are indicated in Figure 1, which was constructed using neighbor-joining \cite{petkov2005evidence}
%LD calculation: estimated LD using D9, the difference between the observed frequency of an allelic combination and its random expectation, relative to the maximum deviation possible given the allele frequencies of the two markers [14,15]. D9 corrects for differences in allele frequencies and describes LD equally well when there is selection for or against the combination of majority alleles. A cumulative Fisher's exact test (FET) was used to compute the probability (pFET) of obtaining an equally or more extreme distribution under the null hypothesis of random allelic association between pairs of SNPs.  \cite{petkov2005evidence}
%Permutaation test: In one set, marker locations were randomized while maintaining the assignments of alleles to strains (Figure 2, red triangles), and in the other set the assignments of alleles to strains were randomized while preserving allele ratios and marker locations (Figure 2, solid circles) \cite{petkov2005evidence}
%It is difficult to escape the conclusion that the selective factors acting to generate LD domains and networks during inbreeding reflect clustering and/or interaction of functionally related elements along chromosomes \cite{petkov2005evidence}

%While these ``long range haplotypes'' can extend over a few hundred kb in unrelated humans [5], they still span only a very small fraction of an entire chromosome. \cite{koch2013long}
%Considerably less attention has been paid to patterns of LD between pairs of sites that are separated by much greater genetic distances (say, 1 cM or more). \cite{koch2013long}
%finding substantial long range linkage disequilibrium (LRLD) suggests that countervailing forces are at work. \cite{koch2013long}
%1) One possibility is population admixture [6], which has been proposed to explain unusual patterns of LRLD in some human populations \cite{koch2013long}
%2) A second contributing force is drift. Even in a population at demographic equilibrium, recombination between distant chromosome blocks will largely but not completely erase LD caused by drift. Recurrent bottlenecks are particularly effective at generating LD [9], and may have contributed importantly to disequilibria in nonAfrican populations of humans \cite{koch2013long}
%3) Third, epistatic selection can maintain linkage disequilibrium indefinitely [11]. Epistasis has been implicated in the LD observed between two pairs of genes in humans [12,13].  \cite{koch2013long}
%4) Fourth, the hitchhiking of linked sites with a positively-selected mutation can generate large haplotype blocks that result in disequilibria over the region that they span [3,4]. \cite{koch2013long}
%5) Fifth, structural variation in chromosomes, such as inversions, can alter patterns of recombination and consequently cause LD to extend over unusually large regions of a chromosome [14-16]. \cite{koch2013long}
%
% to our knowledge there has been only one previous survey of associations between chromosomal regions across the entire human genome using high-density data. Sved [17] studied correlations in heterozygosity between chromosome blocks. His analysis of the HapMap phase 3 data found evidence of associations between blocks at distances of up to 10 cM and weak correlations between blocks on different chromosomes, but he did not attempt to assess their statistical significance. Lawrence et al. [18] provided a web-based tool for exploring long distance linkage disequilibria in the HapMap data, but did not go on to study patterns in the data. \cite{koch2013long}
%This paper investigates patterns of LRLD in the YRI population (the Yoruba in Ibadan, Nigeria) from the HapMap Phase 2 dataset of single nucleotide polymorphisms [23]. YRI also has weaker short-range disequilibria that might otherwise obscure the patterns of LRLD \cite{koch2013long}
%We calculated the disequilibria between all pairs of SNPs on the same chromosome, then analyze these data with new statistical methods.  \cite{koch2013long}
%Using null distributions generated by randomization, we find significant excess of disequilibria on all 22 autosomes in the Yoruba population.  \cite{koch2013long}
%Data: 120 YRI haplotypes that were genotyped at over 2.86106 SNPs in HapMap Phase 2 (data build 22) \cite{koch2013long}
%LD issues: Most commonly used measures of linkage disequilibria are not well suited for that purpose [8]. For example, a large value of D9 is likely to result from sampling if allele frequencies are near 0 or 1, while even a small value is unlikely to appear by chance if allele frequencies are intermediate and the sample size is large. \cite{koch2013long}
%We therefore use the probability that a value of the disequilibrium D as large or larger than that in the sample would be observed if there is no association in the population from which the sample is drawn, conditioned on the sampled allele frequencies at the two loci. This probability, which we denote pD, is given by the tail of Fisher's exact test [8,28,29] \cite{koch2013long}
%As the distance between a pair of sites on a chromosome grows large (specifically, the product of the recombination rate and the effective population size becomes much greater than 1), the sampling distribution for two-locus haplotypes converges on that of Fisher's exact test [30,31].  \cite{koch2013long}

% NOT ADDED
%Patches: When a pair of distant sites are in disequilibrium, it is likely that other sites near to them will also be associated as a result of shortrange associations [17,32]. In effect, the underlying structure in the data is disequilibrium between pairs of chromosomal blocks rather than between pairs of individuals sites \cite{koch2013long}
%To control for this we used a simple and efficient ad hoc strategy that identifies ``patches'' of disequilibria. \cite{koch2013long}
%Results: We take two approaches to search for nonrandom patterns of LRLD. We first ask whether observed values of pD are more extreme than expected. For this purpose we determined the most extreme (that is, smallest) value of pD in each patch, then calculated the mean of these extreme values across all patches on a chromosome. We refer to this statistic as pDmax.  \cite{koch2013long}
%-	Second, we ask whether the number of LRLD patches observed for a given chromosome is greater than expected by chance. We denote this statistic as nP. \cite{koch2013long}
%To test for the statistical significance of pDmax and nP, we generate their null distributions using a randomization method \cite{koch2013long}
%There are two motivations behind this method. First, it preserves the allele frequencies at each site. Second, it maintains the structure of short range disequilibria in the sample.  \cite{koch2013long}
%Computational time: Constructing these null distributions is the most computationally intensive part of our method. For the analyses reported below, over 4.861014 values of pD were computed, and the project consumed about 34,000 hours of CPU time. \cite{koch2013long}
%All of the 22 chromosomes show significant values for pDmax at the p,0.05 level, and all remain significant after a Bonferroni correction for multiple tests. For the second test statistic, n , 19 chromosomes show significant P values, 18 of which remain significant after the Bonferroni correction. These results suggest there is long-range linkage disequilibrium in the YRI population. \cite{koch2013long}
%[LRLD] have been little studied, they may be indicators of important evolutionary processes \cite{koch2013long}

%Whereas most existing epistasis screens explicitly test for a trait, it is also possible to implicitly test for fitness traits by searching for the overor under-representation of allele pairs in a given population.  \cite{ackermann2012systematic}
%Such analysis of imbalanced allele pair frequencies of distant loci has not been exploited yet on a genome-wide scale, mostly due to statistical difficulties such as the multiple testing problem. We propose a new approach called Imbalanced Allele Pair frequencies (ImAP) for inferring epistatic interactions that is exclusively based on DNA sequence information. \cite{ackermann2012systematic}
%Most gene interaction studies explicitly measure a phenotype such as growth rate or viability [ \cite{ackermann2012systematic}
%However, one can also study implicit phenotypes by searching for the overor under-representation of certain allele pairs in a given population. \cite{ackermann2012systematic}
%Such allele pairs are examples of Dobzhansky-Mu ̈ller incompatibilities: they establish a fitness bias in favor of individuals inheriting the over-represented allele combination [15]. In their most extreme form such incompatibilities are embryonic lethal. \cite{ackermann2012systematic}
%In this context, an implicit phenotype is a trait that is not explicitly measured in the sample but whose regulators can still be inferred from the genotype data. \cite{ackermann2012systematic}
%Here, we propose to address this problem by exploiting the additional information gained from studying family trios. We show that by analyzing a sufficiently large number of individuals with known family structure it becomes possible to detect substantially more interactions than what is expected if all markers were independent. \cite{ackermann2012systematic}
%Our method, called ``Imbalanced Allele Pair frequencies (ImAP)'' is based on inspecting 3|3 contingency tables that track the frequencies of all possible two-locus allele combinations in heterozygous individuals (assuming a diploid genome). The test that we propose is similar to a x2 test in that it compares the observed frequencies in this table to expected frequencies assuming independence. However, our version corrects the expected frequencies for confounding factors such as family structure or allelic drift [21]. \cite{ackermann2012systematic}
%In a population of 2,002 heterozygous mice with known family structure genotyped at 10,168 markers we identify 168 LD block pairs with imbalanced alleles \cite{ackermann2012systematic}

%LD: non-physical linkages between different mutations (or single nucleotide polymorphisms, SNPs)  \cite{wang2012ldgidb}
%These interactions can be physical protein interactions, regulatory interactions, functional compensation/antagonization or many other forms of interactions.  \cite{wang2012ldgidb}
%non-physical SNP linkages, coupled with knowledge of SNP-disease associations may shed more light on the role of gene interactions in human disorders. \cite{wang2012ldgidb}
%exonic regions of protein-coding genes from the HapMap database to construct a database named the Linkage-Disequilibrium-based Gene Interaction database (LDGIdb). The database stores 646,203 potential human gene interactions, which are potential interactions inferred from SNP pairs that are subject to long-range strong linkage disequilibrium (LD), or non-physical linkages. To minimize the possibility of hitchhiking, SNP pairs inferred to be non-physically linked were required to be located in different chromosomes or in different LD blocks  \cite{wang2012ldgidb}
%Here we consider only the subpopulations that contain at least 20 individuals. \cite{wang2012ldgidb}
%strong LD (r2 ≥ 0.8); \cite{wang2012ldgidb}

%INBREED MICE: The process of inbreeding to homozygosity imposes intense selective pressures; all efforts among some species have failed, and with mice, only a fraction of initial attempts succeeded.  \cite{petkov2005evidence}
%Accordingly, we can expect that if clustering of functionally related genes is a common feature of mammalian genomes, there is likely to be selection for coadapted allelic combinations among the genes encoding functions that influence fitness and survival during inbreeding. This would result in regions of linkage disequilibrium (LD) among inbred strain genomes; i.e., some allelic combinations should occur more often than expected by chance. \cite{petkov2005evidence}
%Data: 1,456 SNPs, chosen for their high information content, among a set of 60 common and wildderived inbred mouse strains chosen for their genetic diversity.  \cite{petkov2005evidence}
% The identity of these strains and the phylogenetic relationships among them are indicated in Figure 1, which was constructed using neighbor-joining \cite{petkov2005evidence}
%LD calculation: estimated LD using D9, the difference between the observed frequency of an allelic combination and its random expectation, relative to the maximum deviation possible given the allele frequencies of the two markers [14,15]. D9 corrects for differences in allele frequencies and describes LD equally well when there is selection for or against the combination of majority alleles. A cumulative Fisher's exact test (FET) was used to compute the probability (pFET) of obtaining an equally or more extreme distribution under the null hypothesis of random allelic association between pairs of SNPs.  \cite{petkov2005evidence}
%Permutaation test: In one set, marker locations were randomized while maintaining the assignments of alleles to strains (Figure 2, red triangles), and in the other set the assignments of alleles to strains were randomized while preserving allele ratios and marker locations (Figure 2, solid circles) \cite{petkov2005evidence}
%It is difficult to escape the conclusion that the selective factors acting to generate LD domains and networks during inbreeding reflect clustering and/or interaction of functionally related elements along chromosomes \cite{petkov2005evidence}

%Long-range linkage disequilibria (LRLD) between sites that are widely separated on chromosomes may suggest that population admixture, epistatic selection, or other evolutionary forces are at work.  \cite{koch2013long}
%We quantified patterns of LRLD on a chromosome-wide level in the YRI population of the HapMap dataset of single nucleotide polymorphisms (SNPs). \cite{koch2013long}
%We calculated the disequilibrium between all pairs of SNPs on each chromosome (a total of .261011 values) and evaluated significance of overall disequilibrium using randomization.  \cite{koch2013long}
%The results show an excess of associations between pairs of distant sites (separated by .0.25 cM) on all of the 22 autosomes.  \cite{koch2013long}
%Disequilibria between closely-linked sites result largely from random genetic drift or (equivalently) the common ancestry of unrecombined chromosome blocks.  \cite{koch2013long}
%While these ``long range haplotypes'' can extend over a few hundred kb in unrelated humans [5], they still span only a very small fraction of an entire chromosome. \cite{koch2013long}
%Considerably less attention has been paid to patterns of LD between pairs of sites that are separated by much greater genetic distances (say, 1 cM or more). \cite{koch2013long}
%finding substantial long range linkage disequilibrium (LRLD) suggests that countervailing forces are at work. \cite{koch2013long}
%1) One possibility is population admixture [6], which has been proposed to explain unusual patterns of LRLD in some human populations \cite{koch2013long}
%2) A second contributing force is drift. Even in a population at demographic equilibrium, recombination between distant chromosome blocks will largely but not completely erase LD caused by drift. Recurrent bottlenecks are particularly effective at generating LD [9], and may have contributed importantly to disequilibria in nonAfrican populations of humans \cite{koch2013long}
%3) Third, epistatic selection can maintain linkage disequilibrium indefinitely [11]. Epistasis has been implicated in the LD observed between two pairs of genes in humans [12,13].  \cite{koch2013long}
%4) Fourth, the hitchhiking of linked sites with a positively-selected mutation can generate large haplotype blocks that result in disequilibria over the region that they span [3,4]. \cite{koch2013long}
%5) Fifth, structural variation in chromosomes, such as inversions, can alter patterns of recombination and consequently cause LD to extend over unusually large regions of a chromosome [14-16]. \cite{koch2013long}
% to our knowledge there has been only one previous survey of associations between chromosomal regions across the entire human genome using high-density data. Sved [17] studied correlations in heterozygosity between chromosome blocks. His analysis of the HapMap phase 3 data found evidence of associations between blocks at distances of up to 10 cM and weak correlations between blocks on different chromosomes, but he did not attempt to assess their statistical significance. Lawrence et al. [18] provided a web-based tool for exploring long distance linkage disequilibria in the HapMap data, but did not go on to study patterns in the data. \cite{koch2013long}
%This paper investigates patterns of LRLD in the YRI population (the Yoruba in Ibadan, Nigeria) from the HapMap Phase 2 dataset of single nucleotide polymorphisms [23]. YRI also has weaker short-range disequilibria that might otherwise obscure the patterns of LRLD \cite{koch2013long}
%We calculated the disequilibria between all pairs of SNPs on the same chromosome, then analyze these data with new statistical methods.  \cite{koch2013long}
%Using null distributions generated by randomization, we find significant excess of disequilibria on all 22 autosomes in the Yoruba population.  \cite{koch2013long}
%Data: 120 YRI haplotypes that were genotyped at over 2.86106 SNPs in HapMap Phase 2 (data build 22) \cite{koch2013long}
%LD issues: Most commonly used measures of linkage disequilibria are not well suited for that purpose [8]. For example, a large value of D9 is likely to result from sampling if allele frequencies are near 0 or 1, while even a small value is unlikely to appear by chance if allele frequencies are intermediate and the sample size is large. \cite{koch2013long}
%We therefore use the probability that a value of the disequilibrium D as large or larger than that in the sample would be observed if there is no association in the population from which the sample is drawn, conditioned on the sampled allele frequencies at the two loci. This probability, which we denote pD, is given by the tail of Fisher's exact test [8,28,29] \cite{koch2013long}
%As the distance between a pair of sites on a chromosome grows large (specifically, the product of the recombination rate and the effective population size becomes much greater than 1), the sampling distribution for two-locus haplotypes converges on that of Fisher's exact test [30,31].  \cite{koch2013long}
%Patches: When a pair of distant sites are in disequilibrium, it is likely that other sites near to them will also be associated as a result of shortrange associations [17,32]. In effect, the underlying structure in the data is disequilibrium between pairs of chromosomal blocks rather than between pairs of individuals sites \cite{koch2013long}
%To control for this we used a simple and efficient ad hoc strategy that identifies ``patches'' of disequilibria. \cite{koch2013long}
%Results: We take two approaches to search for nonrandom patterns of LRLD. We first ask whether observed values of pD are more extreme than expected. For this purpose we determined the most extreme (that is, smallest) value of pD in each patch, then calculated the mean of these extreme values across all patches on a chromosome. We refer to this statistic as pDmax.  \cite{koch2013long}
%-	Second, we ask whether the number of LRLD patches observed for a given chromosome is greater than expected by chance. We denote this statistic as nP. \cite{koch2013long}
%To test for the statistical significance of pDmax and nP, we generate their null distributions using a randomization method \cite{koch2013long}
%There are two motivations behind this method. First, it preserves the allele frequencies at each site. Second, it maintains the structure of short range disequilibria in the sample.  \cite{koch2013long}
%Computational time: Constructing these null distributions is the most computationally intensive part of our method. For the analyses reported below, over 4.861014 values of pD were computed, and the project consumed about 34,000 hours of CPU time. \cite{koch2013long}
%All of the 22 chromosomes show significant values for pDmax at the p,0.05 level, and all remain significant after a Bonferroni correction for multiple tests. For the second test statistic, n , 19 chromosomes show significant P values, 18 of which remain significant after the Bonferroni correction. These results suggest there is long-range linkage disequilibrium in the YRI population. \cite{koch2013long}
%[LRLD] have been little studied, they may be indicators of important evolutionary processes \cite{koch2013long}


\subsection{Epistasis \& GWAS}

In recent years there have been a growing number of GWAs, most of them have used a single-locus analysis strategy, in which each variant is tested individually for association with a specific phenotype \cite{cordell2009detecting}
It may be inadequate to describe complex disease the relationship between genotype and phenotype by simply summing the modest effects from several contributing loci \cite{culverhouse2002perspective}.
The extent to which epistasis is involved in complex traits is not known so we cannot assume that epistasis will be found for every trait in every population. \cite{carlborg2004epistasis}
However epistasis has been overlooked and that it should to be routinely explored in complex trait studies.  \cite{carlborg2004epistasis}.
This is particularly important for researchers of moderately heritable complex diseases for which locus-by-locus analyses have not accounted for the predicted genetic variance should pursue a hypothesis of epistatic loci \cite{culverhouse2002perspective} that owing to their interaction, might not be identified by using standard single-locus tests \cite{cordell2009detecting}.
It is also hoped that detecting such interactions will allow to elucidate biological pathways that underpin disease \cite{cordell2009detecting}.

Failure to detect epistasis does not rule out it's presence \cite{zuk2012mystery}.
On the one hand the reason why complex human phenotypes fail to find evidence for epistatic interactions may simply be that analytic methods inherently exclude epistasis \cite{culverhouse2002perspective}.
On the other hand, individual interaction effects are expected to be much smaller than linear effects, and the sample size required to detect an effect scales inversely with the square of the effect size. 
As an example provided by \cite{zuk2012mystery} consider two variants with frequency $20\%$ and increasing risk by 1.3 fold, which is a large effect.
In such case, assuming $50\%$ power, significance level $5 \times 10^{-8}$  and equal number of cases and controls; the sample size required for single loci analysis would be $4,900$.
In comparisson, the sample size required to detect pairwise interaction between those two variants using the same power and an appropriately corrected significance level is roughly $450,000$, so a researcher studying $100,000$ samples would discover all single acting loci but would find little evidence of epistatic interactions, which may be the reason why geneticists that have tested for pairwise epistasis between loci have found few significant signals \cite{zuk2012mystery}.
It should be noted that even though GWAs involving over $500,000$ samples are not available at the moment, studies using sample sizes in this order are expected to become available within the next couple of years.

Existing approaches for searching interactions can be grouped into five broad categories. \cite{li2011detecting}:
\begin{enumerate}
	\item \textbf{Exhaustive search} methods use classical statistics such as the Pearson's $\chi^2$ test or logistic regression that are natural extensions of methods commonly used for single-locus tests in GWAS. 
It should be noted that the number of tests necessary to evaluate all two-way, three-way and four-way interactions for 30-60 candidate loci, has a range similar to the number of tests suggested for a single GWAS, thus searching for n-way interactions among all the markers would be impracticable. \cite{culverhouse2002perspective}.
Approaches developed to detect epistasis: combinatorial partitioning method \cite{REF}, restricted partitioning method \cite{REF}, multifactor-dimensionality reduction \cite{REF}, multivariate adaptive regression spline \cite{REF}, logistic regression methods and backward genotype-trait association (BGTA)\cite{REF}. 
Unfortunately even though many of these look promising, many of them have only been tested on small data sets \cite{zhang2007bayesian}.
Furthermore, methods based on brute-force searches such as (combinatorial partitioning method and multifactor-dimensionality reduction) are impractical for large data sets \cite{zhang2007bayesian}.
Nevertheless it was shown \cite{li2011detecting} that it can be feasible to perform GWAS level analysis in some cases and that simple methods explicitly considering interactions can actually achieve reasonably high power with realistic sample sizes under different interaction models with some marginal effects, even after adjustments of multiple testing using the Bonferroni correction.
	
	\item \textbf{Linkage disequilibrium} methods use patterns in disease population under two-locus disease models \cite{zhao2006test} association can be estimated assuming that deviation of the penetrance from independence at an individual locus creates linkage disequilibrium (LD) even if two loci are unlinked.  \cite{zhao2006test}
Under the assumptions that two disease-susceptibility loci are in Hardy-Weinberg equilibrium (HWE) and are unlinked
they show that in the presence of interaction the two loci will be in linkage disequilibrium in the disease population \cite{zhao2006test}.
They develop a test statistic for detection deviations from LD, intuitively they test interaction by comparing the difference in the LD levels between two unlinked loci between cases and controls \cite{zhao2006test}.
Since the frequency of a haplotype is equal to the product of the frequencies of the component alleles of the haplotype, thus in the absence of interaction the proportion of individuals carrying a haplotype in the disease population is equal to the product of the proportions of individuals carrying the component alleles of the haplotype in the disease population \cite{zhao2006test}.
They further show that under the null hypothesis, this test statistic asymptotically converges to a central $\chi^2$ distribution.
In their power comparison simulations they show that in general this LD-based test statistic has much smaller p-values than those of logistic regression analysis \cite{zhao2006test} conlcuding that their test has much higher power than alternatives such as logistic regression \cite{zhao2006test}.

	\item \textbf{Stochastic search} methods use sampling to infer whether a locus is a risk locus, jointly affects disease, or a background locus.
A Bayesian approach for genome-wide case-control studies denoted `bayesian epistasis association mapping' (BEAM) \cite{zhang2007bayesian} is a representative example of this type of methods.
 BEAM treats the disease-associated markers and their interactions via a bayesian partitioning model and computes the posterior probability that each marker (using Markov chain Monte Carlo).
The method uses predictors in the form of genetic marker loci divided into three groups: i) markers not associated with disease, ii) markers individually contributing to disease risk, and iii) markers that interact \cite{zhang2007bayesian}.
Membership of each marker in each of the three groups is defined by the prior distributions.
Given a prior distributions for regression coefficients values given by group membership, a posterior distribution can be generated using MCMC simulation. \cite{cordell2009detecting}.
At the end, it uses a special statistic (B-Statistic) to infer significance from the samples in MCMC. 
Although it avoids computing all interactions but theoretically could find high-order interactions. 
Since the method was originally designed for genotypes markers, its power can be hampered by allele frequency discrepancies between unobserved disease loci and associated genotyped marker/s\cite{zhang2007bayesian}.
This is a common problem when using indirect markers and the authors show that in an extreme case when the MAF discrepancy was maximized all tested methods had little power to detect interaction associations \cite{zhang2007bayesian}.
In the original paper, the authors apply BEAM to a data set containing $116,204$ SNPs genotyped for $96$ affected individuals and $50$ controls. for an association study of age-related macular degeneration (AMD).
Unfortunately BEAM did not find any significant interactions \cite{zhang2007bayesian} most likely due to the small sample size.
Runtime and power are primarily determined by the number of MCMC rounds with a suggested number of MCMC iteration as the quadratic of the number of SNPs, thus limiting the applicability \cite{li2011detecting}.
So it cannot be applied to large GWAS studies because computational limitations make it unsuitable to handle over $500,000$ markers with sample sizes of $5,000$ or more individuals which are now commonly sequenced or genotyped \cite{cordell2009detecting}.

	\item \textbf{Conditional search} methods usually perform analyses in stages \cite{li2011detecting}.
A small subset of significant loci is identified in the first stage, typically using single locus association statistics.
Then this subset is mined using multi-locus association using an exhaustive method. 
A well known approach in this category is ``stepwise logistic regression" which works as follows: (i) all markers are individually tested for associations with disease and ranked; (ii) the top (usually $10\%$) are selected, 3) all two-way (or three-way) interactions are tested and ranked for associations. 
Even this stepwise approach can become computationally intractable for high-order interactions \cite{zhang2007bayesian}.
Analysis of stepwise logistic regression approach to identify two-way and three-way interactions demonstrated that searching for interactions in genome-wide association mapping can be more fruitful than traditional approaches that exclusively focus on marginal effects \cite{zhang2007bayesian}.
As a counter argument for stepwise logistic regression, we should take into account that in the presence of epistasis the effect of one locus is altered or masked by another locus, thus power to detect the first locus is likely to be reduced and the joint effects will be hindered by their interaction \cite{cordell2002epistasis}. 
Methods based on conditional search can greatly reduce the computational burden by a couple of orders of magnitude, but with the risk of missing markers with small marginal effect \cite{li2011detecting}.

	\item \textbf{Machine learning} approaches can also be used to infer epistasis.
A popular approach uses Random Forests \cite{li2011detecting} or other regression trees partitioning approaches based on classification.
In this context, trees are constructed using rules based on the values of a predictor variable such as a SNP to differentiate observations such as case-control status \cite{cordell2009detecting}.
A popular rule selection mechanism is to use the variable that maximizes the reduction in Gini impurity \cite{REF} at each node (intuitively, when child nodes have lower impurity from a split based on an attribute each child node will have purer classification).
Random forests are constructed by drawing samples with replacement from the original sample. 
A classification tree is created for each bootstrap sample, but only a random subset of the possible predictor variables is considered. 
This results in a `forest' of trees have been trained on a particular sample of observations. \cite{cordell2009detecting}
Instead of trying to create a monolithic learner, this type of methods called ``ensemble systems" attempt to create many heterogeneous ``weak" (or simple) learners. 
The outcomes of these heterogeneous systems are combined to create an improved model \cite{li2011detecting}.

In an extension of AdaBoost algorithm, the authors incorporates an importance score based on Gini impurity to select candidate SNP \cite{li2011detecting} in a way that genotype frequencies from the two classes (case and control) are expected to be more different.
Decision trees are usually built with binary splits, but since genotype data can be ${0, 1, 2}$, they also extended their method to create a ternary split
AdaBoost draws bootstrap samples to increase the power of a weak learner by weighting the individuals when bootstrapping. 
So when a weak learner misclassifies an individual, the weight of that individual is increased, as a consequence hard to classify individuals are more likely to be included in future bootstrap samples. 
The ensemble votes by weighting weak learner instances by training set accuracy.  \cite{li2011detecting}.
Using simulation, they claim that their method outperforms not only a similar ensemble approaches, but also several statistical methods, although thei mention performance degradation when the risk allele frequency is low \cite{li2011detecting}.

\end{enumerate}

%Bayesian model selection methods offer an alternative approach for selecting predictor variables and the interactions between them that are the best predictors of phenotype. The key difference between Bayesian model selection and simple comparisons of nested regression models using frequentist (non-Bayesian) procedures is the specification of prior distributions for the unknown regression parameters as well as for a dimension parameter in a Bayesian approach. This dimension parameter specifies how many non-zero predictors are included \cite{cordell2009detecting}
%A posterior distribution for these parameters, given the observed data, can then be calculated using Markov chain Monte Carlo (MCMC)93 simulation techniques, in which one traverses the space of the possible models (sets of parameter values), sampling the outputs of the simulation run at intervals. Although MCMC is a flexible approach, it can require some care with respect to the choice of prior distributions, proposal schemes (determining how one moves between models) and the number of iterations required to achieve convergence. \cite{cordell2009detecting}

%The main reason that most studies of complex human phenotypes fail to find evidence for epistatic interactions may simply be that commonly used designs and analytic methods inherently minimize or exclude the possibility of epistasis (Frankel and Schork 1996) \cite{culverhouse2002perspective}
%The complex relationship between genotype and phenotype, however, may ultimately prove to be inadequately described by simply summing the modest effects from several contributing loci. \cite{culverhouse2002perspective}
%We note that the number of tests necessary to evaluate all two-, three-, and four-way interactions, for 30-60 candidate loci, has a range similar to the number of tests suggested for a single genomewide association scan using SNPs (Collins et al. 1999; Kruglyak 1999) \cite{culverhouse2002perspective}
%Thus, although searching for two-, three-, four-, or n-way interactions among all the markers in a genome scan would not be practicable, a candidate-locus approach based on a genome scan for linkage may be. \cite{culverhouse2002perspective}


%Following the identification of several disease-associated polymorphisms by genome-wide association (GWA) analysis, interest is now focusing on the detection of effects that, owing to their interaction with other genetic or environmental factors, might not be identified by using standard single-locus tests \cite{cordell2009detecting}
%...it is hoped that detecting interactions between loci will allow us to elucidate the biological and biochemical pathways that underpin disease.  \cite{cordell2009detecting}
%In recent years, the field has been revolutionized by the success of genome-wide association (GWA) studies1-5. Most of these studies have used a single-locus analysis strategy, in which each variant is tested individually for association with a specific phenotype \cite{cordell2009detecting}
%However, a reason that is often cited for the lack of success in genetic studies of complex disease6,7 is the existence of interactions between loci.  \cite{cordell2009detecting}
%If a genetic factor functions primarily through a complex mechanism that involves multiple other genes and, possibly, environmental factors, the effect might be missed if the gene is examined in isolation without allowing for its potential interactions with these other unknown factors. \cite{cordell2009detecting}

%Several approaches have been developed to detect epistasis, including the combinatorial partitioning method (CPM)7, the restricted partitioning method (RPM)8, multifactor-dimensionality reduction (MDR)2, multivariate adaptive regression spline (MARS)9, the logistic regression method10 and backward genotype-trait association (BGTA)11. Although these methods all showed promise, they have been tested only on small data sets.  \cite{zhang2007bayesian}
%	methods based on brute-force searches such as CPM and MDR are impractical for large data sets \cite{zhang2007bayesian}
%	STEPWISE LOGISTIC REGRESSION: The stepwise logistic regression approach of ref. 12 works as follows: (i) all markers are individually tested and ranked for marginal associations with the disease; (ii) the top 10\% of markers are selected, among which all k-way (k 1⁄4 2 or 3) interactions are tested and ranked for associations. The authors of ref. 12 also proposed an exhaustive logistic regression testing approach, which we choose not to consider in this study because of its prohibitive computational cost.  Note that even their stepwise approach can become computationally intractable for high-order interactions.  \cite{zhang2007bayesian}
%	Recently, a simulation study12 explored the use of a stepwise logistic regression approach to identify two-way and three-way interactions. The authors demonstrated that searching for interactions in genome-wide association mapping can be more fruitful than traditional approaches that exclusively focus on marginal effects. \cite{zhang2007bayesian}

%For complex traits such as diabetes, asthma, hypertension and multiple sclerosis, the search for susceptibility loci has, to date, been less successful than for simple Mendelian disorders. This is probably due to complicating factors such as an increased number of contributing loci and susceptibility alleles, incomplete penetrance, and contributing environmental effects \cite{cordell2002epistasis}
%The presence of epistasis is a particular cause for concern, since, if the effect of one locus is altered or masked by effects at another locus, power to detect the first locus is likely to be reduced and elucidation of the joint effects at the two loci will be hindered by their interaction.  \cite{cordell2002epistasis}
%Although genetic interactions are hard to detect in humans (see below), several cases involving variants with large marginal effects have been recently reported in Hirschsprung's disease, ankylosing spondylitis, psoriasis, and type I diabetes  \cite{zuk2012mystery}
%...geneticists have tested for pairwise epistasis between loci, but have found few significant signals. \cite{zuk2012mystery}

%...The reason is that individual interaction effects are expected to be much smaller than linear effects, and the sample size required to detect an effect scales inversely with the square of the effect size. If n loci had equivalent effects, the sample size to detect the n loci would thus scale with $n^2$, whereas the sample size to detect their $n^2$ interactions scales with $n^4$. \cite{zuk2012mystery}
%Suppose that we consider two variants with frequency 20\% that contribute to different pathways and increase risk by 1.3-fold (which is a large effect relative to those typically seen in GWAS). The sample size required to detect the variants is 4,900 (with 50\% power and genome-wide significance level of $\alpha = 5 \times 10^{-8}$ in a genome-wide association study with an equal number of cases and controls), whereas the sample size required to detect their pairwise interaction is roughly 450,000 (at 50\% power and an appropriate significance level to account for multiple hypothesis testing). A researcher who studied 100,000 samples would likely discover all of the loci but would find little evidence of epistatic interactions. \cite{zuk2012mystery}
%In short, the failure to detect epistasis does not rule out the presence of genetic interactions sufficient to cause substantial phantom heritability \cite{zuk2012mystery}

%Cases only. The most straightforward multilocus analysis of cases-only data is a $\chi^2$ test of independent segregation for the loci.  \cite{culverhouse2002perspective}
%Case-control. A second approach is a multilocus case-control analysis. One method for doing this would be to compare the distribution of cases among the 3L genotypes, where L is the number of biallelic loci being simultaneously examined, versus the distribution of controls. In this analysis, a sample of N cases and N unrelated controls drawn from a population modeled by table 3 will, again, yield an expected $\chi^2$ statistic 2N. However, the degrees of freedom under the null hypothesis are now 8. \cite{culverhouse2002perspective}

%We developed a general theory for studying linkage disequilibrium (LD) patterns in disease population under two-locus disease models.  \cite{zhao2006test}
%Our results showed that the P values of the LD-based statistic were smaller than those obtained by other approaches, including logistic regression models. \cite{zhao2006test}
%This was further developed by Cockerham4 and Kempthorne5 into the modern representation that treats statistical gene interactions as interaction terms in a regression model or a generalized linear model on allelic effects.2,6-11 \cite{zhao2006test}
%we propose to define interaction between two unlinked loci (or genes) for a qualitative trait as the deviance of the penetrance for a haplotype at two loci from the product of the marginal penetrance of the individual alleles that span the haplotype.  \cite{zhao2006test}
%DEFINE: Deviance \cite{zhao2006test}
%Interaction between two unlinked loci will result in deviation of the penetrance of the two-locus haplotype from independence of the marginal penetrance of the alleles at an individual locus, which in turn will create linkage disequilibrium (LD) even if two loci are unlinked.  \cite{zhao2006test}
%Therefore, it is possible to develop statistics for detection of interaction between two unlinked loci by use of deviations from LD \cite{zhao2006test}
%we assume that two disease-susceptibility loci are in Hardy-Weinberg equilibrium (HWE) and are unlinked.  \cite{zhao2006test}
%[they show that] Under this definition, in the absence of interaction, two unlinked loci in the disease population will be in linkage equilibrium \cite{zhao2006test}
%Similar to linkage equilibrium, where the frequency of a haplotype is equal to the product of the frequencies of the component alleles of the haplotype, absence of interaction between two unlinked loci implies that the proportion of individuals carrying a haplotype in the disease population is equal to the product of the proportions of individuals carrying the component alleles of the haplotype in the disease population \cite{zhao2006test}
%TEST STATISTIC:  \cite{zhao2006test}
%	Intuitively, we can test interaction by comparing the difference in the LD levels between two unlinked loci between cases and controls \cite{zhao2006test}
%	We can show that test statistic TI is asymptotically distributed as a central x2 distribution under the (1) null hypothesis of no interaction between two unlinked loci  \cite{zhao2006test}
%we compared the power of the LD-based statistic with that of the logistic model.  \cite{zhao2006test}
%Power comparison with logistic regression analysis demonstrated that this LD-based test statistic has much higher power in detecting interaction than does the logistic regression method. \cite{zhao2006test}
%To further evaluate its performance for detection of interaction between two loci, the proposed LD-based statistic was applied to two published data sets. Our results showed that, in general, P values of the test statistic TI were much smaller than those of other approaches, including logistic regression analysis. \cite{zhao2006test}


%Although some existing computational methods for identifying genetic interactions have been effective for small-scale studies, we here propose a method, denoted `bayesian epistasis association mapping' (BEAM), for genome-wide case-control studies \cite{zhang2007bayesian}
%BEAM treats the disease-associated markers and their interactions via a bayesian partitioning model and computes, via Markov chain Monte Carlo, the posterior probability that each marker set is associated with the disease.  \cite{zhang2007bayesian}
%In the past century, scientists have made great progresses in mapping genes responsible for mendelian diseases. However, genetic variants underlying most common (or `complex') diseases are non-mendelian. \cite{zhang2007bayesian}
%These variants are typically not rare in the population (42%). They show very little effect independently with low penetrance, but they may interact with each other in complex ways. \cite{zhang2007bayesian}
%It has been speculated that epistasis ubiquitously contributes to complex traits partly because of the sophisticated regulatory mechanisms encoded in the human genome1.  \cite{zhang2007bayesian}
%EPI EXAMPLES: An increasing number of reports have indicated the presence of multilocus interactions in many human complex traits, such as breast cancer2, post-PTCA stenosis3, essential hypertension4, atrial fibrillation5 and type 2 diabetes6. \cite{zhang2007bayesian}
%GWAS EPISTASIS [Discussion]: We also applied BEAM to an association study of age-related macular degeneration (AMD)13, which included B100,000 SNP markers. Although BEAM did not find significant interactions in the AMD data set, it was able to discover two-way or three-way interactions among the B100,000 SNPs simulated based on the AMD data. \cite{zhang2007bayesian}
%BEAM METHOD:
%	The BEAM algorithm takes case-control genotype marker data as input and produces, via MCMC simulations, posterior probabilities that each marker is associated with the disease and involved with other markers in epistasis.  \cite{zhang2007bayesian}
%	The input genotyped markers should be in their natural genomic order when there is linkage disequilibrium (LD) among some of them. The method can be used either in a `pure' bayesian sense or just as a tool to discover potential `hits'. For the former, one relies on the reported posterior probabilities to make inferential statements; as for the latter, one can take the reported hits and use another procedure to test whether these hits are statistically significant.  \cite{zhang2007bayesian}
%	The latter approach is more robust to model selection and prior assumptions (such as Dirichlet priors with arbitrary parameters) and is less prone to the slow mixing problem in the MCMC computational procedure. We also propose the B statistic to facilitate the latter approach and show that it is more powerful than the standard w2 statistic for epistasis detections. \cite{zhang2007bayesian}
%	For the non-epistasis model (model 1), all three epistasis mapping methods performed similarly to the single-marker w2 test (Fig. 1), indicating that the power for detecting marginal associations was not compromised by using the more complex models. \cite{zhang2007bayesian}
%	Notably, results for model 4 suggest that stepwise methods can miss markers with small or no marginal effects, whereas BEAM can get these markers back through iterations. \cite{zhang2007bayesian}
%POWER ISSUES RELATED TO AF: 
%	The power of association mapping can be greatly hampered by the discrepancy of allele frequencies between unobserved disease loci and associated genotyped markers15 \cite{zhang2007bayesian}
%	For data sets with large MAF discrepancies and moderate LD, the power of all methods suffered.  \cite{zhang2007bayesian}
%	At the extreme case when the MAF discrepancy was maximized (that is, MAF 1⁄4 0.5), all methods had little power in detecting interaction associations \cite{zhang2007bayesian}
%	The impact of LD on power seemed to be less profound than the effect of MAF discrepancy.  \cite{zhang2007bayesian}
% ANALISYS:
%	DATA: The data set contains 116,204 SNPs genotyped for 96 affected individuals and 50 controls. \cite{zhang2007bayesian}
%	RESULTS: BEAM found no significant interactions associated with AMD from this data set. It is possible that the small sample size of 146 individuals is insufficient for detecting subtle epistasis interactions. \cite{zhang2007bayesian}


%The purpose of this Review is to provide a survey of the methods and related software packages that are currently being used to detect the interactions between the genetic loci that contribute to human genetic disease. \cite{cordell2009detecting}
%Interaction as departure from a linear model. The most common statistical definition of interaction relies on the concept of a linear model that describes the relationship between an outcome variable and a predictor variable or variables \cite{cordell2009detecting}
%Arguably the most well-known form of this type of analysis is simple linear or least squares regression26, in which we relate an observed quantitative outcome y (for example, weight) to a predictor variable x (for example, height) using a `best fit' line or regression \cite{cordell2009detecting}
%From a statistical point of view, interaction represents departure from a linear model that describes how two or more predictors predict a phenotypic outcome \cite{cordell2009detecting}
%For a disease outcome and case-control data, rather than modelling a quantitative trait y, the usual approach is to model the expected log odds of disease as a linear function of the relevant predictor variables \cite{cordell2009detecting}
%DEFINITION Penetrance: The probability of displaying a particular phenotype (for example, succumbing to a disease) given that one has a specific genotype. \cite{cordell2009detecting}
%DEFINITION: Marginal effects: The average effects (for example, penetrances) of a single variable, averaged over the possible values taken by other variables. These could be calculated for one locus of a two-locus system as the average of the two-locus penetrances, averaged over the three possible genotypes at the other locus. \cite{cordell2009detecting}
%For or simplicity, I have concentrated here on defining interaction in relation to two genetic factors (two-locus interactions). In practice, however, for complex diseases we might also expect three-locus, four-locus and even higher-level interactions. Mathematically, such higherlevel interactions are simple extensions to the two-locus models described earlier.  \cite{cordell2009detecting}
%CASE ONLY METHODS:
%	A case-only test of interaction can therefore be performed by testing the null hypothesis that there is no correlation between alleles or genotypes at the two loci in a sample that is restricted to cases alone. This test can easily be performed using a simple χ2 test of independence between genotypes (a four degrees of freedom test) or alleles (a one degree of freedom test), or using logistic or multinomial regression in any statistical analysis package. \cite{cordell2009detecting}
%	The main problem with the case-only test is its requirement that the genotype variables are not correlated in the general population. It is this assumption, rather than the design per se, that provides the increased power compared with case-control analysis \cite{cordell2009detecting}
%	The caseonly test is therefore unsuitable for loci that are either closely linked or show correlation for another reason (for example, if certain genotype combinations are related to viability). \cite{cordell2009detecting}
%Tests for association allowing for interaction: From a mathematical point of view, a test for association at a given locus C while allowing for interaction with another locus B (a joint test16) corresponds to comparing the fit to the observed data of a linear model in which the main effects of B, C and their interactions are included  \cite{cordell2009detecting}
%Theoretically, if no interaction effects exist, these joint tests will be less powerful than marginal singlelocus association tests. However, if interaction effects exist, then the power of joint tests can be higher than that of single-locus approaches52. \cite{cordell2009detecting}
%CLASSIFICATION TREE: Recursive partitioning approaches are based on classification and regression trees111. Trees are constructed (see the figure) using rules that determine how wella split at a node (based on the values of a predictor variable such as a SNP) can differentiate observations with respect to the outcome variable (such as case-control status). A popular splitting rule is to use the variable that maximizes the reduction in a quantity known as the Gini impurity111,112 at each node.  \cite{cordell2009detecting}
%RAMDOM FOREST: A random forest is constructed by drawing with replacement several bootstrap samples of the same size (for example, the same number of cases and controls) from the original sample. An unpruned classification tree is grown for each bootstrap sample, but with the restriction that at each node, rather than considering all possible predictor variables, only a random subset of the possible predictor variables is considered. This procedure results in a `forest' of trees, each of which will have been trained on a particular bootstrap sample of observations. \cite{cordell2009detecting}
%BAYESIAN MODEL SELECTION: Bayesian model selection techniques92 offer an alternative approach for selecting predictor variables and the interactions between them that are the best predictors of phenotype. The key difference between Bayesian model selection and simple comparisons of nested regression models using frequentist (non-Bayesian) procedures is the specification of prior distributions for the unknown regression parameters as well as for a dimension parameter in a Bayesian approach. This dimension parameter specifies how many non-zero predictors are included \cite{cordell2009detecting}
%	A posterior distribution for these parameters, given the observed data, can then be calculated using Markov chain Monte Carlo (MCMC)93 simulation techniques, in which one traverses the space of the possible models (sets of parameter values), sampling the outputs of the simulation run at intervals. Although MCMC is a flexible approach, it can require some care with respect to the choice of prior distributions, proposal schemes (determining how one moves between models) and the number of iterations required to achieve convergence. \cite{cordell2009detecting}
%	BEAM: Bayesian Epistasis Association Mapping. A recently proposed MCMC approach that is specifically designed to detect interacting, as well as non-interacting, loci is Bayesian epistasis Association Mapping13, which is implemented in the software package BeAM. In BeAM, predictors in the form of genetic marker loci are divided into three groups: group 0 contains markers that are not associated with disease, group 1 contains markers that contribute to disease risk only by main effects and group 2 contains markers that interact to cause disease by a saturated model. Given prior distributions that describe the membership of each marker in each of the three groups and prior distributions for the values of the relevant regression coefficients given group membership, a posterior distribution for all relevant parameters can be generated using MCMC simulation. In addition to making inferences in a fully Bayesian inferential framework, one can use the results from BeAM in a frequentist hypothesistesting framework by calculating a `B-statistic'13 that tests each marker or set of markers for significant association with a disease phenotype. \cite{cordell2009detecting}
%	EBAM LIMITATIONS: BeAM cannot currently handle the 500,000-1,000,000 markers that are now routinely being genotyped in genome scans of 5,000 or more individuals. \cite{cordell2009detecting}

%We extend the basic AdaBoost algorithm by incorporating an intuitive importance score based on Gini impurity to select candidate SNPs.  \cite{li2011detecting}
%Permutation tests are used to control the statistical significance. \cite{li2011detecting}
%We have performed extensive simulation studies using three interaction models to evaluate the efficacy of our approach at realistic GWAS sizes, and have compared it with existing epistatic detection algorithms. \cite{li2011detecting}
%CURRENT METHODS: Generally speaking, existing approaches for searching gene- gene or SNP-SNP interactions can be grouped into four broad categories. \cite{li2011detecting}
%	1) Methods in the first category rely on exhaustive search. Classical statistics such as the Pearson's χ2 test or the logistic regression that are commonly used as single-locus tests for GWAS can potentially be used in searching for pairwise interactions. Marchini et al. (2005) have shown that explicitly modeling of interactions between loci for GWAS with hundreds of thousands of markers is computationally feasible. They also showed that these simple methods explicitly considering interactions can actually achieve reasonably high power with realistic sample sizes under different interaction models with some marginal effects, even after adjustments of multiple testing using the Bonferroni correction. \cite{li2011detecting}
%	2) The second category consists of methods relying on stochastic search, with BEAM (Zhang and Liu, 2007) as one representative of such algorithms. Later algorithms in this category [e.g. epiMODE (Tang et al., 2009)] largely adopted and extended BEAM. BEAM uses Markov chain Monte Carlo (MCMC) sampling to infer whether each locus is a disease locus, a jointly affecting disease locus, or a background (uncorrelated) locus. The algorithm begins by assigning each locus to each group according to a prior distribution. Using the Metropolis-Hastings algorithm, it attempts to reassign the group labels to each locus. At the end, it uses a special statistic, called the B-Statistic, to infer statistical significance from the hits sampled in MCMC. This approach avoids computing all interactions, but can still theoretically find high-order interactions. The number of MCMC rounds is the primary parameter that mediates runtime, as well as power. The suggested number of MCMC rounds is in the quadratic of the number of SNPs, which limits applicability of BEAM on large datasets. \cite{li2011detecting}
%	3) Methods in the third category are machine learning approaches such as tree-based methods or support vector machines (SVM). For example, a popular ensemble approach, Random Forests  \cite{li2011detecting}
%	4) Methods in the forth category rely on conditional search. In such a case, analyses are performed in stages (Evans et al., 2006; Li, 2008). A small subset of promising loci is identified in the first stage, normally using single locus methods, and multi-locus methods are used in the later stage(s) to model interactions based on the selection in the first stage. Stepwise regression has been widely used in this case and several different strategies have been studied in the literature. Methods based on conditional search can greatly reduce the computational burden by a couple of orders of magnitude, but with the risk of missing markers with small marginal effect. One should also notice that the conditional search category is more like a strategy rather than an approach. In addition to single-locusbased methods, any approaches discussed previously, especially the machine learning ones, can be used to search for candidates in the first stage. \cite{li2011detecting}
%THIS METHOD: We extend the basic AdaBoost algorithm by incorporating an intuitive importance score based on Gini impurity to select candidate SNP \cite{li2011detecting}
%Instead of trying to create a monolithic learner or model, ensemble systems attempt to create many heterogeneous versions of simpler learners, called weak learners. The opinions of these heterogeneous experts are then combined to formulate a complete picture of the data.  \cite{li2011detecting}
%Usually, a SNP is selected to ensure largest homogeneity in the child nodes. In our implementation, we use the gain on Gini Impurity. Intuitively, when child nodes have lower impurity from a split based on an attribute (i.e. a SNP here), each child node will have purer classification. Therefore, the genotype frequencies from the two classes (case and control) are expected to be more different.  \cite{li2011detecting}
%Usually decision trees are built with binary splits, where individuals with one value of the feature are placed into one group, and the remainder into the other. Since genotype data is three valued, we extend this to do a ternary split. \cite{li2011detecting}
%Despite only using marginal effects to select SNPs, decision trees can still detect some interaction. Because of the recursive partitioning, lower nodes are effectively conditioned on the value of their parents.
%The core idea of AdaBoost is to draw bootstrap samples to increase the power of a weak learner. This is done by weighting the individuals when drawing the bootstrap sample. When a weak learner instance misclassifies an individual, the weight of that individual is increased (and increased more if the weak learner instance was otherwise accurate). Thus, hard to classify individuals are more likely to be included in future bootstrap samples. In the end, the ensemble votes for class labels weighting the weak learner instances by training set accuracy.  \cite{li2011detecting}

\subsection{Epistasisi GWAS: Power issues}


%===================================================================================================
% REWRITE
%===================================================================================================

 We have seen that, if the true genetic model underlying a disease is purely epistatic, with no additive or dominance variation at any of the susceptibility loci, then association methods analyzing one locus at a time will have no power to detect the loci.  \cite{culverhouse2002perspective}
First, we expect that, with a sufficient number of contributing loci, purely epistatic interactions could account for virtually all the variation in affection status for diseases with any prevalence \cite{culverhouse2002perspective}
Of course, there are subclasses of purely epistatic models (providing no marginal evidence for the involvement of any single locus) for which, in addition, no two, three, or L1 loci jointly give evidence of involvement in the disorder. This leads to the concern that even assessment of all two-, three-, and (L1)-way interactions among candidate loci may be insufficient for detection of the contributing loci. \cite{culverhouse2002perspective}
The restriction on maximum heritabilities in these models is most easily seen by examining L-locus models for which no collection of L 􏰂 1 loci shows marginal deviations.  \cite{culverhouse2002perspective}

A small number of recent studies have explored this idea for the genome-level identification of epistatic interactions: if a large number of individuals is genotyped at a large number of genomic positions, it becomes possible to test all allele pairs for overand underrepresentation in that population [18-20]. \cite{ackermann2012systematic}
However, even though some methodological progress has been made [18], previous studies could hardly identify a significant number of interactions. The main obstacle is the humongous number of statistical hypotheses tested when comparing all markers in a genome against all markers. \cite{ackermann2012systematic}

QTL: We present FastEpistasis, an efficient parallel solution extending the PLINK epistasis module, designed to test for epistasis effects when analyzing continuous phenotypes. \cite{schupbach2010fastepistasis}
FastEpistasis is capable of testing the association of a continuous trait with all single nucleotide polymorphism (SNP) pairs from 500 000 SNPs, totaling 125 billion tests, in a population of 5000 individuals in 29, 4 or 0.5 days using 8, 64 or 512 processors. \cite{schupbach2010fastepistasis}
It tests epistatic effects in the normal linear regression of a quantitative response on marginal effects of each SNP and an interaction effect of the SNP pair, where SNPs are coded as additive effects, taking values 0,1 or 2. The test for epistasis reduces to testing whether the interaction term is significantly different from zero. \cite{schupbach2010fastepistasis}
The computations are based on applying the QR decomposition to derive least squares estimates of the interaction coefficient and its standard error.  \cite{schupbach2010fastepistasis}

\subsection{Epistatic GWAS \label{sec:epigwas}}

% GWAS & Epistasis
Genome wide association studies have traditionally focused on single variants or nearby groups of variants. An often cited reason for the lack of discovery of high impact risk factors in complex disease is that these models ignore loci interactions \cite{cordell2009detecting} which have recently been pointed out as a potential solution for the ``missing heritability" problem \cite{zuk2012mystery, zuk2014searching}. With interactions being so ubiquitous in cell function, one may wonder why they have been so neglected by GWAS. There are several reasons: i) models using interactions are much more complex \cite{gao2010classification} and by definition non-linear, ii) information on which proteins interacts with which other proteins is incomplete \cite{venkatesan2009empirical}, iii) in the cases where there protein-protein interaction information is available, precise interacting sites are rarely known \cite{venkatesan2009empirical}. Taking into account the last two items, we need to explore all possible loci combinations, thus the number of $N$ order interactions grows as $O(M^N)$ where $M$ is the number of variants \cite{de2013emerging}. This requires exponentially more computational power than single loci models. This also severely reduces statistical power, which translates into requiring larger cohort, thus increasing sample collection and sequencing costs \cite{de2013emerging}.

In Chapter \ref{ch:gwas} we develop a computationally tractable model for analysing putative interaction of pairs of variants from GWAS involving large case / control cohorts of complex disease. Our model is based on analysing cross-species multiple sequence alignments using a co-evolutionary model in order to obtain informative interaction prior probabilities that can be combined to perform GWAS analysis of pairs of non-synonymous variants that may interact.

%=====================================================================================================
% END OF ADD SECTION
%=====================================================================================================

The definition of epistasis from a statistical perspective is a ``departure from a linear model" \cite{cordell2009detecting}. This means that in a logistic regression model the input for sample $s$ includes terms with each of the genotypes at loci $i$ and $j$), as well as an ``interaction term" $g_{s,i} \cdot g_{s,j}$ \cite{cordell2002epistasis}. 

\begin{eqnarray*} \label{eq:gwasLogRegH1}
    P( d_s | g_{s,i},g_{s,j}) & = & \phi[ \theta_0 + \theta_1 g_{s,i} + \theta_2 g_{s,j} + \theta_3 (g_{s,i} g_{s,j}) \\
    & & ... + \theta_4 c_{s,1} + ... + \theta_m c_{s,N_{cov}} ] \\
\end{eqnarray*}

where $d_s$ is disease status, $\phi(\cdot)$ is the sigmoid function, $c_{s,1}, c_{s,2}, ... $ are covariates for sample $s$.

Models involving interactions between more than two variants can be defined similarly, but require more parameters and extremely large samples are required to accurately fit them.

Several families of approaches for epistatic GWAS exist. Here we mention a few:

\begin{itemize}

\item Allele frequency: In \cite{ackermann2012systematic}, an analysis of imbalanced allele pair frequencies is performed under the assumption that an implicit test for fitness can be achieved looking for over/under-represented allele pairs in a given population. In another study \cite{zhao2006test} the authors infer that interactions can create LD in disease population under two-loci model, then they show how LD-based p-values can uncover interaction and sometimes (in their simulations) outperform logistic regression tests.

\item Bayesian model: In \cite{zhang2007bayesian}, a ``Bayesian partitioning model" is used by providing Dirichlet prior distributions for each partition and computing posterior probabilities using Markov chain Monte Carlo (MCMC) algorithms.  The methodology first test individual makers and picks only the top 10\% to further investigate for epistasis, because it is prohibitive to test all loci.

\item Machine learning: From a machine learning point of view, finding interacting variants is simply an \textit{``optimisation procedure is to find a set of parameters that allows the machine-learning model to most accurately predict class membership (e.g. affected vs unaffected)"} \cite{mckinney2006machine}. Several approaches have emerged to tackle the ``interaction problem" and used a variety of different techniques \cite{koo2013review, mckinney2006machine} , such as neural networks, cellular automata, random forests, multifactor dimensionality reduction, support vector machines, etc.

\end{itemize}

Although all these models have advantages under some assumptions, none of them seems to be a ``clear winner" over the rest \cite{cordell2009detecting}. All of these models suffer from the increase in number of tests that need to be performed, which raises two issues: i) multiple testing, which is often resolved by stringent significance threshold, and ii) computational feasibility, which is solved by efficient algorithms, parallelization, and heuristic approaches to quickly discard uninformative loci combinations. So far, no method for epistatic GWAS has been widely adopted and there is need of different approaches to be explored. In Chapter \ref{ch:gwas} we propose an approach to combine co-evolutionary models and GWAS epistasis of pairs of putatively interacting loci.

%------------------------------------------------------------
% From: A Review for Detecting Gene-Gene Interactions Using Machine Learning Methods in Genetic Epidemiology (2013)
% \cite{koo2013review}
%	Topic: GWAS EPISTASIS METHODS (MACHINE LEARNING)
%
%		EPISTASIS TYPES: 
%			Moreover, there are various types of gene-gene interactions which are syntheticinteraction, epistatic interaction, and suppressive-interaction which are shown in Figure 1. These interactions are particularly important due to the effect of a gene on individual phenotype is depending on more than one additional genes \cite{koo2013review}
%			For instances, synthetic-interaction between two genes is that genes A and B are on different parallel pathways that can obtain the purple phenotype C. If either of the genes is knockout, the purple phenotype C still can be viewed. However, if both of the genes are knockout, it will result in a nonpurple phenotype.  \cite{koo2013review}
%			Next, the example of epistatic-interaction that is the wild type holds a mixed purple and green phenotype of genes C and D. A gene knockout of gene B cannot obtain a purple phenotype of gene C, but green phenotype of gene D still can be seen. A gene knockout of gene A cannot obtain the green and purple phenotypes.  \cite{koo2013review}
%			Furthermore, the example of suppressive-interaction is wild type phenotype showing a purple phenotype since gene A suppresses gene B and gene C is active. A gene knockout of gene B has no effect for result purple phenotype. A knockout of gene A results in a nonpurple phenotype since gene B is still suppressing gene C and if both of the genes A and B are knockout will result in wild type phenotype. \cite{koo2013review}
%
%		ADD ALL METHODS FROM 
%			Table 1: Summary of detect gene-gene interaction using neural network method.
%				COLUMN 1
%				No. Author
%				(1) Ritchie et al. [11]
%				(2) Tomita et al. [12]
%				Keedwell and Narayanan [13]
%				Motsinger et al. [14]
%				Ritchie et al. [15]
%				Motsinger-Reif et al. [16]
%				[17]
%				[18]
%				[4]
%				
%				COLUMN 2
%				Dataset Epistasimodel.
%				Childhood allergic asthma (CAA).
%				Artificial data experiments, rat spinal cord and yeast Saccharomyces Cerevisiae cell cycle.
%				Parkinson's disease.
%				Alzheimer's disease, breast's disease, colorectal disease, and prostate's disease.
%				Epitasis model.
%				Two-locus disease models, multiplicative and epistasis model.
%				Simulated human.
%				Genetic models.
%				
%				COLUMN 3
%				Description
%				GPNN and BPNN were used to model gene-gene interactions by using simulated data. The simulated data contains functional SNPs and nonfunctional SNPs which model the interaction between genes. \cite{koo2013review}
%				Artificial neural network was utilized with parameter decreasing method in order to analyse susceptible SNPs among the Japanese people. \cite{koo2013review}
%				Genetic algorithm which was implemented along with neural networks discovers gene-gene interactions in temporal gene expression dataset by elucidating the information between regulatory connections and interactions between genes, proteins, and other gene products. \cite{koo2013review}
%				GPNN had been used to optimize the architecture of neural network. This method can be used to enhance the identification of gene combinations associated with Parkinson's disease. \cite{koo2013review}
%				GPNN had been used to detect gene-gene interactions and gene-environment interaction in studies of human disease to optimize the architecture of Neural Network by using simulated dataset. \cite{koo2013review}
%				GENN was utilized to discover gene-gene interactions that caused are by noise (for instance, genotyping error, missing data, phenocopy, and genetic heterogeneity) in high dimensional genetic epidemiological data. \cite{koo2013review}
%				NN had been used in simulation study to model the different kind of two-locus disease model by constructing six neural networks. \cite{koo2013review}
%				ATHENA had been used to discover the gene-gene interactions that influence complex human traits by integrating alternative tree-based crossover, back propagation, and domain knowledge in ATHENA. \cite{koo2013review}
%				QTGENN had applied GENN methods to quantitative traits in various types of simulated genetic models. This method had been successfully applied in single-locus models and two-locus models. \cite{koo2013review}
%
%
%			ADD ALL: Table 2: Summary of detect gene-gene interaction using support vector machine method.
%				No. Author
%				Matchenko-
%				(1) Shimko and Dube
%				[23]
%				(2) Chen et al. [19]
%				(3) O ̈ zgu ̈ r et al. [24]
%				(4) Shen et al. [25]
%				(5) Ban et al. [26]
%				(6) Missiuro [21]
%				(7) Fang and Chiu [27]
%				(8) Zhang et al. [28] Marvel and
%				(9) Motsinger-Reif [29]
%				
%				Dataset
%				Simulated disease.
%				Real prostate cancer genotyping.
%				Prostate cancer.
%				Parkinson disease.
%				Type 2 diabetes mellitus-related genes.
%				Caenorhabditis elegans. COGA (genetics of
%				alcoholism). Human cancer.
%				Disease model, M1 and M2.
%				
%				Description
%				Both SVM and artificial neural network (ANN) were used to preselect the combination of SNP to test the importance of potential interactions between genes in complex disease. \cite{koo2013review}
%				SVM was applied in different kinds of combinatorial optimization methods which were recursive feature addition, recursive feature elimination, local search, and genetic algorithm. \cite{koo2013review}
%				Automatic method that was proposed to extract known genes-disease and infer unknown gene-disease association by using automatic literature mining based on dependency parsing and support vector machines. \cite{koo2013review}
%				Authors had employ two-stage method by using SVM with L1 penalty to detect gene-gene interactions for human complex disease. \cite{koo2013review}
%				SVM was used to predict the importance of gene-gene interactions in T2D in the studies of Korean cohort studies. \cite{koo2013review}
%				SVM was utilized in this research to detect interactions between gene in kinase families for Caenorhabditis elegans organism. \cite{koo2013review}
%				SVM-based PGMDR was introduced to study the interactions of gene-gene and gene-covariate in the presence or absence of main effects of genes. \cite{koo2013review}
%				Binary matrix shuffling filter (BMSF) as an efficient SVM search schemes was integrated with SVM to classify cancer tissue samples. \cite{koo2013review}
%				GESVM was applied in large dataset to select important features, parameters, or kernel in SVM. \cite{koo2013review}
%
%
%			ADD ALL: %				Table 3: Summary of detect gene-gene interaction using random forest method.
%				No. Author
%				(1) Lunetta et al. [33]
%				(2) Jiang et al. [34]
%				(3) Schwarz et al. [35]
%				(4) Liu et al. [36]
%				(5) Winham et al. [32]
%				(6) Pan et al. [37]
%				(7) Staiano et al. [38] Chen and Ishwaran
%				
%				Dataset
%				H2M2, H4M2, H8M2, H16M2, H4M4, and H8M4.
%				Three simulated disease model.
%				Crohn's disease.
%				NARAC1 and NARAC2.
%				Five models.
%				Bladder cancer.
%				Familial combined hyperlipidemia (FCH).
%				
%				Description
%				RF as a screening procedure to identify top-ranked true-associated SNPs which can cause disease without losing any interactions. \cite{koo2013review}
%				RF is used to recognize the cases that were against controls and to obtain the Gini importance which is used to measure the contribution of each SNP to the classification performance. \cite{koo2013review}
%				A new method of RJ based on basis RF knowledge was developed to facilitate a fast processing in the high-dimensional of genome-wide analysis data of gene-gene interactions. \cite{koo2013review}
%				RF is used to detect contributed gene-gene interactions for identifing RA susceptibility and to identify SNPs of RA patients to classify them into anticyclic citrullinated protein positive and healthy controls. \cite{koo2013review}
%				Focus on identifing rarely gene-gene interactions and detecting gene-gene interaction effects and their potential effectiveness on high-dimensional data using RF. \cite{koo2013review}
%				The proposed method of MINGRF is proposed to improve the performance of RF such as accuracy and computational time. \cite{koo2013review}
%				RF is used to identify gene-gene interactions that are involved in FCH. FCH increase the plasma triglycerides and/or total cholesterol level of patients and hence increase the risk of coronary heart disease. \cite{koo2013review}
%				RSF as new hunting pathway to detect gene correlation and genomic interactions from a high-dimensional genomic data. \cite{koo2013review}
%
%			PROS AND CONS FOR EACH MOTHOD: ADD ALL!
%				Table 4: Strengths and weaknesses of neural networks, support vector machine, and random forests methods for detect gene-gene interactions. \cite{koo2013review}
%				
%				Methods
%				Neural network
%				Support vector machine (SVM)
%				Random forest (RF)
%				Random jungle (RJ)
%				
%				Strengths
%					Neural network
%					(i) NN is able to model the relationship between disease and single nucleotide polymorphism (SNP) \cite{koo2013review}
%					(ii) NN can make prediction on data where the disease outcome is unknown by learning the outcome given on a dataset (iii) NN is a method that can deal with large volumes of data \cite{koo2013review}
%					(iv) NN is suitable for genetic heterogeneity, high phenocopy rates, polygenic inheritance, and incomplete penetrance. \cite{koo2013review}
%					(v) GPNN and GENN are able to optimize the architecture of NN and possess high power to discover the presence of nonfunctional SNPs. \cite{koo2013review}
%					(vi) GPNN does not overfitting the data (vii) GPNN possesses high power in dealing with epitasis model with weak marginal effect \cite{koo2013review}
%					(viii) GENN outperform GPNN by optimiz NN in fewer generations \cite{koo2013review}
%					(ix) GENN possesses high power to detect high risk loci in complex disease \cite{koo2013review}
%
%					Support vector machine (SVM)
%					(i) SVM can deal with high dimension data set \cite{koo2013review}
%					(ii) SVM can be utilized to classify complex biological gene expression data (iii) Does not trap at local minima \cite{koo2013review}
%					(iv) Not prone to overfitting \cite{koo2013review}
%					(v) SVM is robust to noise \cite{koo2013review}
%					(vi) The output of SVM is more interpretable if compared to MDR (vii) Does not require user-defined decisions for classification \cite{koo2013review}
%					(viii) SVM is ready to be generalized to new structures \cite{koo2013review}
%
%					Random forest (RF)
%					(i) RF does not exhibit strong main effects which uncover interactions among genes. (ii) RF does not ``overfit" the data. \cite{koo2013review}
%					(iii) SNPs predictive of a phenotype are identifying by RF. \cite{koo2013review}
%
%					Random jungle (RJ)
%					(i) RJ is able to analyze data on a genome-wide scale. \cite{koo2013review}
%					(ii) RJ has more computationally efficient than RF. \cite{koo2013review}
%
%			Weaknesses
%					Neural network
%					(i) Presence of black box \cite{koo2013review}
%					(ii) Difficult to list out all possible NN architecture and it causes the difficulty to find the optimal architecture \cite{koo2013review}
%					(iii) GPNN needed parallel processing environment \cite{koo2013review}
%					(iv) GPNN causes the high false positive rate to occur in three locus models \cite{koo2013review}
%					(v) The output of GPNN is binary expression, and it can be hard to interpret (for instance, up to 500 nodes) \cite{koo2013review}
%					(vi) Result of NN was hard to interpret due to the dimensionality problem \cite{koo2013review}
%					(vii) NN needs comprehensive cross-validation to confirm validity \cite{koo2013review}
%
%					Support vector machine (SVM)
%					(i) Presence of black box \cite{koo2013review}
%					(ii) SVM is restricted to pairwise classification \cite{koo2013review}
%					(iii) SVM cannot be directly used for feature selection \cite{koo2013review}
%					(iv) Result produced may be affected by the presence of missing data \cite{koo2013review}
%					(v) The power of SVM might reduce with the presence of genetic heterogeneity \cite{koo2013review}
%					(vi) Additional training maybe needed to correct the bias of prediction accuracy. However, it could be computationally expensive for the proposed procedure (vii) Accuracy produced by SVM might be suboptimal due to the SVM parameter C is forced to be one constant. Hence, a grid search for the parameter is needed by utilizing some promising SNP combinations in order to refine the results. \cite{koo2013review}
%
%					Random forest (RF) / Random jungle (RJ)
%					(i) Presence of black box \cite{koo2013review}
%					(ii) RF does not succeed in GWAS data. (iii) Sometimes RF is underestimating important scores of SNPs without marginal effects. \cite{koo2013review}
%					(iv) RF only detects interactions with large effect size. \cite{koo2013review}
%					If the main effects are weak, RJ fails to detect interactions. \cite{koo2013review}
