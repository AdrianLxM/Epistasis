%%
\newtheorem{lemma}{{\bf Lema}}[chapter]
\newtheorem{theorem}{{\bf Teorema}}[chapter]
\newtheorem{corollary}{{\bf Corolario}}[theorem]
\newtheorem{definition}{{\bf Definici\'on}}[chapter]
\newtheorem{propo}{{\bf Proposicion}}[chapter]

%%
\renewcommand{\vec}[1]{\mbox{$\,${\bf #1}}}
\newcommand{\C}{\mbox{$\mbox{l}\!\!\!\mbox{C}$}}
\newcommand{\N}{\mbox{$\mbox{I}\!\mbox{N}$}}
\newcommand{\R}{\mbox{$\mbox{I}\!\mbox{R}$}}
\newcommand{\Z}{\mbox{$\mbox{Z}\!\!\mbox{Z}$}}
\newcommand{\Rn}{\mbox{$\R^n\,$}}
\newcommand{\Rnxn}{\mbox{$\R^{n \times n}$}}
\newcommand{\Rnxm}{\mbox{$\R^{n \times m}$}}
\newcommand{\Rmxn}{\mbox{$\R^{m \times n}$}}
\newcommand{\Rmxm}{\mbox{$\R^{m \times m}$}}
\newcommand{\inertia}[1]{\mbox{$\cal I$($#1$)}}
\newcommand{\ceil}[1]{\mbox{$\left\lceil #1 \right\rceil$}}
\newcommand{\floor}[1]{\mbox{$\left\lfloor #1 \right\rfloor$}}
\newcommand{\ceilfrac}[2]{\mbox{$\left\lceil\frac{#1}{#2}\right\rceil$}}
\newcommand{\floorfrac}[2]{\mbox{$\left\lfloor\frac{#1}{#2}\right\rfloor$}}
\newcommand{\GEP}{\mbox{(\ref{eq:GEP})}}
\newcommand{\AMB}{\mbox{$A-\mu B$}}
\newcommand{\M}{\mbox{$\cal M$}}
\newcommand{\B}{\mbox{$\cal B$}}
\newcommand{\Q}{\mbox{$\cal Q$}}
\newcommand{\D}[1]{\mbox{${\cal D}_#1$}}
\newcommand{\U}{\mbox{$\cal U$}}
\newcommand{\V}{\mbox{$\cal V$}}
\newcommand{\lead}[2]{\mbox{$#1_{[#2]}$}}
\newcommand{\minor}[2]{\mbox{$\det{\lead{#1}{#2}}$}}
\newcommand{\inner}[3]{\mbox{$<\!\!#1,#2\!\!>_{#3}$}}
\newcommand{\ie}{i.e.\ }
\newcommand{\eg}{e.g.\ }

% Define \mytable
% Include one of my tables, in the standard way
%  Parm 1 is the table name
%  Parm 2 is the caption for the "List of tables"
%  Parm 3 is the real caption
\newcommand{\mytable}[3]{
    \begin{table}[htbp]
        \begin{minipage}{\textwidth}
          \begin{center}
	          \TableCaptionOpt{#2 \label{tab:#1}}{#3}
			  \label{tab:#1}
              \input{tables/#1/table.tex}
          \end{center}
        \end{minipage}
    \end{table}
    \normalsize
}

% Define \fig
% Include one of my figures, in the standard way
%  Parm 1 : File name (no extension)
%  Parm 2 : Label (without 'fig:')
%  Parm 3 : Width
%  Parm 4 : Caption
%  Parm 5 : Figure name (for table of contents)

% \newcommand{\fig}[5]{
% \begin{figure}[ht]
%     \begin{center}
%         \begin{minipage}{#3}
%             \includegraphics[ width={#3} ]{figs/#1}
%             \FigureCaptionOpt{#4}{#5}
%             \label{fig:#2}
%         \end{minipage}
%     \end{center}
% \end{figure}
% }

\newcommand{\fig}[5]{
\begin{figure}
  \begin{center}
    \includegraphics[ width={#3} ]{figs/#1}
    \FigureCaptionOpt{#4}{#5}
    \label{fig:#2}
  \end{center}
\end{figure}
}

% Define \fig
% Include one of my figures (but the figure actually contains a table)
%  Parm 1 : File name (no extension)
%  Parm 2 : Width
%  Parm 3 : Caption
%  Parm 4 : Table name (for table of contents)
\newcommand{\figtab}[5]{
\begin{figure}
  \renewcommand{\figurename}{Table }
  \begin{center}
    \includegraphics[ width={#3} ]{figs/#1}
    \TableCaptionOpt{#4}{#5}
    \label{tab:#2}
  \end{center}
\end{figure}
}

